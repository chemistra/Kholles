\begin{exercise}{Ivre, il chute dans un ravin}{2}{Spé}
{Diffusion thermique}{bermu}

On souhaite isoler un bâtiment d'habitation. On dispose pour cela de trois matériaux :
\begin{itemize}
    \item Le béton ($\lambda = 1.75$ W.m$^{-1}$.K$^{-1}$)
    \item Le plâtre ($\lambda = 1.50$ W.m$^{-1}$.K$^{-1}$)
    \item La laine de verre ($\lambda = 0.04$ W.m$^{-1}$.K$^{-1}$)
\end{itemize}

\begin{questions}
    \questioncours Flux d'énergie surfacique, loi de Fourier.
    \question Lequel de ces matériaux peut-être considéré comme un isolant thermique ?
    \question On décide de fabriquer le bâtiment avec des murs uniquement en béton (épaisseur 10 cm) recouvert du côté intérieur par une couche de plâtre (épaisseur 2 cm). Quelle puissance surfacique passe à travers des murs ainsi-conçu, pour une température intérieure de 20 degrés et extérieure de 5 degrés ?
    \question Que devient cette puissance si on ajoute 10 cm de laine de verre entre le béton et le plâtre ? Ce procédé est-il intéressant ?
    \question Donner la température moyenne du béton, de la laine de verre et du plâtre dans le dernier cas. 
\end{questions}

\end{exercise}
