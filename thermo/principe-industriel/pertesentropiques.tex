\begin{exercise}{Pertes entropiques}{2}{Sup, spé}
{Thermodynamique, Premier principe industriel}{lelay}

On considère un compresseur qui reçoit en entrée un débit de 16 m$^3$/s d'air à $T_1 = 300$ K et à pression atmosphérique. En sortie l'air est compressé à 7 bars

\begin{questions}
    \questioncours Thermodynamique des systèmes ouverts
    \question Quelle serait la température en sortie si le compresseur fonctionnait de manière isentropique ? Quelle serait alors la puissance fournie par le compresseur ?
    \uplevel{En réalité, on mesure la température de sortie à $T_2 = 530$ K.}
    \question Quelle est la puissance réellement fournie par le compresseur ? Expliquer l'origine des pertes.
    \question Quel est le rendement isentropique du compresseur ? On note $\eta$ le rendement isentropique, qui est le ratio de la puissance réellement fournie sur la puissance minimale nécessaire (fonctionnement isentropique).
    \question Calculer l'entropie massique créée lors de la compression du gaz.
\end{questions}

\end{exercise}