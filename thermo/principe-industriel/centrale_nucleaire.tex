\begin{exercise}{Centrale nucléaire du Blayais  }{2}{Sup, spé}
{Thermodynamique, Machines thermiques, Premier principe industriel}{bermudez,uhl2019-1}

La centrale nucléaire du Blayais, suitée dans l'estuaire de la Gironde, à 50 km en aval de Bordeaux, est constituée de 4 réacteurs à eau pressurisée (REP --- 70 bars, 350$^\circ$ C), refroidis par l'eau de l'estuaire qui est pompée via des canalisations sous-marines.

\textsf{Question ouverte : } \`A l'aide des données suivantes :
\begin{questions}
    \question Estimez le rendement de la centrale nucléaire. Commenter.
    \question Comparez ce rendement avec celui de Carnot ainsi que le rendrement à puissance maximale $\eta_\text{pmax} = 1 - \sqrt{T_\text{f}/T_\text{c}}$.
    \question Estimez l'échauffement de l'eau de l'estuaire en aval de la centrale. La législation impose un échauffement maximal de $+3^\circ$ C ; est-elle respectée ?
\end{questions}

% Please add the following required packages to your document preamble:
% \usepackage{multirow}
\begin{table}[H]
\begin{tabularx}{\linewidth}{llCCCrrr}
\multicolumn{1}{c}{\multirow{2}{*}{\textsfbf{\begin{tabular}[c]{@{}c@{}}Nom du\\ réacteur\end{tabular}}}} & \multicolumn{1}{c}{\multirow{2}{*}{\textsfbf{Modèle}}} & \multicolumn{3}{c}{\textsfbf{Capacité {[}MW{]}}}                                                                                      & \multicolumn{1}{c}{\multirow{2}{*}{\textsfbf{\begin{tabular}[c]{@{}c@{}}Début\\ constr.\end{tabular}}}} & \multicolumn{1}{c}{\multirow{2}{*}{\textsfbf{\begin{tabular}[c]{@{}c@{}}Raccord\\ au réseau\end{tabular}}}} & \multicolumn{1}{c}{\multirow{2}{*}{\textsfbf{\begin{tabular}[c]{@{}c@{}}Mise en\\ service\end{tabular}}}} \\
\multicolumn{1}{c}{}                                                                                    & \multicolumn{1}{c}{}                                 & \textsfbf{thermique} &\textsfbf{brute} & \textsfbf{nette} & \multicolumn{1}{c}{}                                                                                  & \multicolumn{1}{c}{}                                                                                      & \multicolumn{1}{c}{}                                                                                            \\ \hline
Blayais-1                                                                                            & CP1                                                  & 2 785                                         & 951                                      & 910                                      & janvier 1977                                                                                          & juin 1981                                                                                                 & déc. 1981                                                                                                   \\
Blayais-2                                                                                              & CP1                                                  & 2 785                                         & 951                                      & 910                                      & janvier 1977                                                                                          & juillet 1982                                                                                              & fév. 1983                                                                                                    \\
Blayais-3                                                                                              & CP1                                                  & 2 785                                         & 951                                      & 910                                      & avril 1978                                                                                            & août 1983                                                                                                 & nov. 1983                                                                                                   \\
Blayais-4                                                                                              & CP1                                                  & 2 785                                         & 951                                      & 910                                      & avril 1978                                                                                            & mai 1983                                                                                                  & oct. 1983                                                                                                      
\end{tabularx}
\end{table}

\paragraph{Données :} débit moyen annuel de la Gironde à Blayes $\SI{950}{}$ m$^3\cdot$s$^{-1}$.

\end{exercise}

\begin{solution}

    \begin{questions}
        \question $\eta = \dfrac{\text{P brute}}{\text{P thermique}} = 35\%$
        \question Si on prend $T_f = 25^\circ$ C, l'eau de la Gironde en hiver, et $T_c = 350^\circ$ soit un rendement de Carnot de 52\% et de Curzon de 35\%
        \question La puissance envoyée dans l'eau est environ $P_f = 4\times (2785 - 951) = 7,4$ GW. On a donc
        $$\delta Q = \dd{m} c_m \Delta T \qqtext{soit} \Delta T = \dfrac{P_f}{D_v \rho c_m}$$

        AN :         $\Delta T = 1,8^\circ$ C.

        On peut donner $c = 4,18$ J/g/K si il y a un trou de mémoire
    \end{questions}
\end{solution}