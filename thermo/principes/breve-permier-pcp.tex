\begin{exercise}{Freinage}{2}{Sup, Spé}
{Thermodynamique,Premier principe,\'Energie interne}{bermudez,bedo}

Une Renault Twingo se déplace à grande vitesse sur une route départementale. Elle freine brusquement et s'arrête.

\begin{questions}
    \questioncours Capacité thermique. Cas des gaz parfaits et des phases condensées.
    \question \textsf{Question ouverte} Estimer l'élévation de température des plaquettes de frein de la voiture suite au freinage. \`A partir de quelle vitesse les freins à disque risquent de ne plus fonctionner correctement ? Est-il pertinent de piler sur les freins passer une certaine vitesse ?
\end{questions}

\paragraph{Données}~(toutes ne sont pas utiles)
\begin{itemize}
    \item Renault Twingo I (2007--2012), 800 kg, $1.4\times1.6\time 3.4$ m, vitesse maximale 170 km/h, 60 ch. max.
    \item Acier inox : densité 8, capacité thermique $\mathrm{500 J\cdot kg^{-1}\cdot K^{-1}}$, point de fusion 1400$^\circ$C
    \item Incandescence des métaux : chauffé au rouge 500$^\circ$C, chauffé à blanc 1200$^\circ$C.  
\end{itemize}

\end{exercise}

\begin{solution}

$\dfrac{1}{2}M v^2 = m c \Delta T$, $M$ étant la masse de la voiture et $m$ celle de la plaquette de freins.

En odg, les plaquettes freins font quelques centimètres et ont une masse d' 1kg environ.

Donc : $\Delta T \ [^\circ\text{C}] = v^2/20 [\mathrm{km/h}]$.

Pour 30 km/h, $\Delta T = 45^\circ$C

On passe au rouge pour 100 km/h, $\Delta T = 500^\circ$C

On passe au blanc pour 150 km/h, $\Delta T = 1200^\circ$C

On passe le point de fusion pour 170 km/h, $\Delta T = 1200^\circ$C

\textsf{Limites :} dissipation de la chaleur par l'air, frottements aérodynamiques de l'air, conduction de la chaleur dans la carrosserie via les essieux.

\end{solution}