\begin{exercise}{Transition de phase --- Idées reçues}{0}{Sup, Spé}
{Thermodynamique,Changement d'état}{bermudez}

\begin{questions}
    \questioncours Corrigez les affirmations ci-dessous
    \begin{parts}
        \part Un changement d'état à lieu à température et pression constantes.
        \part Au cours d'un changement d'état isotherme d'un corps pur, l'énergie interne ne varie pas.
        \part Au cours de la vaporisation isotherme d'un corps pur, la variation d'énergie interne est égale à la chaleur latente de vaporisation.
        \part Au point triple, l'état du système est parfaitement déterminé.
    \end{parts}
\end{questions}

\end{exercise}

\begin{solution}

\begin{questions}
    \questioncours ~
    \begin{parts}
        \part La variance est de 2, donc \emph{imposer la pression et la quantité de matière du système revient à imposer également la température} : la relation $P(T,n)$ est fixée, mais pas $(P,T,n)$ individuellement.
        \part Elle varie
        \part C'est la variation d'enthalpie qui est égale à la chaleur latente.
        \part Au point triple, la variance est de 1, donc $P(n),T(n)$ sont fixés \emph{seulement si $n$ est fixé, i.e.} si le système est fermé.
    \end{parts}
\end{questions}

\end{solution}