\begin{exercise}{Cloche à vide}{1}{Sup}
{Thermodynamique}{lelay}

On considère une cloche cylindrique de hauteur $h$ et de rayon $R$.

\begin{questions}
    \questioncours Définition cinétique de la pression.
    \uplevel{Initialement, la cloche est posée sur une table, elle est à l'équilibre avec l'environnement extérieur qui se trouve dans les CNTP.}
    \question Quelle est la quantité de gaz contenu dans la cloche ?
    \uplevel{On fait le vide dans la cloche jusqu'à ce qu'il ne reste plus que $n$ moles de gaz. On suppose que la cloche est adiabate.}
    \question Quelle est la pression à l'intérieur de la cloche ?
    \question Quelle force faut-il exercer sur la cloche pour réussir à la soulever ?
    \question Que devient cette force lorsque $n$ tend vers 0 ?
    \question Même question (avec $n = 0$) dans le cas d'une cloche conique de même rayon et d'angle au sommet $\alpha$
    \question Même question dans le cas d'une cloche hémisphérique
    \question Quid du cas général ?
\end{questions}

\end{exercise}