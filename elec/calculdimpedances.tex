\begin{exercise}{Calculs d'impédance}{1}{Sup}
{\'Electrocinétique}{lelay}

\begin{questions}
    \questioncours Loi de composition des impédances et des admittances
    \question Quelle est l'impédance équivalente du circuit suivant ?
\begin{circuit}
      \draw
      (0,0) to [R, l^=$R$] (2,0) 
            to [L, l^=$C$] (4,0)
            to [C, l^=$C$] (6,0);
\end{circuit}
    \question Donner, pour chacun des circuits suivants, le rapport $s/e$
\end{questions}

\begin{circuit}
      \draw
      (0,0) to [open, v_=$e$, *-o] (0,2)
      (0,2) to [R, l=$R$] (2,2) 
      to [C, l^=$C$, *-*] (2,0)
      (2,2) to [short] (4,2)
      (4,0) to [open, v_=$s$, *-o] (4,2)
      (0,0) to [short] (4,0);
\end{circuit}

\begin{circuit}
      \draw
      (0,0) to [open, v_=$e$, *-o] (0,2)
      (0,2) to [R, l=$R$] (2,2) 
      to [C, l^=$C$, *-*] (2,0)
      (2,2) to [R, l=$R$] (4,2)
      to [C, l^=$C$, *-*] (4,0)
      (4,2) to [short] (6,2)
      (6,0) to [open, v_=$s$, *-o] (6,2)
      (0,0) to [short] (6,0);
\end{circuit}

\begin{circuit}
      \draw
      (0,0) to [open, v_=$e$, *-o] (0,2)
      (0,2) to [R, l=$R$] (2,2) 
      to [C, l^=$C$, *-*] (2,0)
      (2,2) to [R, l=$R$] (4,2)
      to [C, l^=$C$, *-*] (4,0)
      (4,2) to [R, l=$R$] (6,2)
      to [C, l^=$C$, *-*] (6,0)
      (6,2) to [short] (8,2)
      (8,0) to [open, v_=$s$, *-o] (8,2)
      (0,0) to [short] (8,0);
\end{circuit}

\end{exercise}

\begin{solution}

avec $p = jRC\omega = j\frac{\omega}{\omega_0}$
\begin{align*}
    \frac{s}{e} &= \frac{1}{1+p} \\
    \frac{s}{e} &= \frac{1}{(1+p)(2+p)-1} \\
    \frac{s}{e} &= \frac{1}{(1+p)(2+p)^2-1} \\
\end{align*}
\end{solution}