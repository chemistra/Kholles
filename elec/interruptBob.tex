% Niveau :      PCSI
% Discipline :  Elec
% Mots clés :   Elec, Ordre 1

\begin{exercise}{Bobine avec interrupteur}{1}{Sup}
{\'Electrocinétique, Circuits d'ordre 1}{lelay}

On cherche à décrire le circuit suivant :
\begin{circuit}
      \draw
      node [ground] at (0,0) {}
      to [vsource, v^>=$E$] (0,2)
      to [R, l^=$R$] (2,2)
      to [nos, l=$K$] (4,2)
      to [R, l^=$R$] (6,2)
      to [vsource, v^<=$E$] (6,0)
      node [ground]{}
      (2,2) to [R, l_=$R$] (2,0)
      node [ground]{}
      (4,2) to [L, l^=$L$] (4,0)
      node [ground]{};
\end{circuit}

\begin{questions}
    \questioncours Donne la loi reliant la tension aux bornes d'une bobine au courant qui la traverse. Unité et ordre de grandeur de l'inductance d'une bobine.
    \question Quelle est la tension aux bornes de la bobine tant que l'interrupteur reste ouvert ? Le courant qui la traverse ?
    \question À $t = 0$, on ferme l'interrupteur $K$. Donner le circuit équivalent à $t = 0^+$ et à $t = \infty$.
    \question Donner l'équation différentielle vérifiée par la tension $u$ aux bornes de la bobine.
    \question Quel est le temps caractéristique d'évolution de $u(t)$ ? Quel nom proposez-vous de lui donner ?
    \question Écrire $u(t)$ et donner l'allure de sa courbe.
\end{questions}
\end{exercise}


\begin{solution}

\begin{questions}
    \questioncours ok
    \question $E$
    \question $t = 0$, la bobine est un interrupteur ouvert, $t=\infty$, le condensateur est un fil
    \question $3L \dv{u}{t} + R u = 0$
    \question $\tau = 3L/R$
    \question Une exponentielle décroissante pépère.
\end{questions}

\end{solution}