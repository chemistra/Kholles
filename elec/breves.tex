% Niveau :      PCSI - PC
% Discipline :  Elec
% Mots clés :   Elec, Ordre 2

\begin{exercise}{Brèves}{1}{Sup,Spé}
{\'Electrocinétique, Circuits d'ordre 2}{bermu}

\begin{questions}
    \question Trois lampes à incandescence de puissances différentes (par exemple 25 W , 60 W et 100 W) sont reliées en
série ; l'ensemble est branché sur un générateur délivrant une tension de 220 V. \\
En admettant que
l'éclairement d'une lampe est proportionnel à la puissance électrique qu'elle reçoit, quelle lampe brillera le
plus fort ? Quid si les lampes sont en dérivation ? \\
N.B. : Ce qu'on appelle la puissance d'une lampe est sa puissance lorsqu'elle est soumise à une tension
caractéristique soit ici 220 V.

    \question Un poste de distribution EDF assimilé à une fem $E$ constante est relié à une installation domestique par deux
fils très longs dont la résistance totale inévitable est $r$. On appelle $P_f$ la puissance fournie par le poste de
distribution, $P_u$ la puissance utile reçue par l'installation et $\rho$ le rendement de l'opération. \\
Exprimer $\rho$ en fonction de $E$, $r$ et $P_f$ . En déduire que, pour une puissance fiéee, le rendement est d'autant
meilleur que $E$ est grand. Quelle est l'application pratique de ce résultat ?
\end{questions}
\end{exercise}