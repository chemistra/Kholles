% Niveau :      PCSI
% Discipline :  Elec
% Mots clés :   Elec, Ordre 1

\begin{exercise}{Combinaison de dipôles}{1}{Sup}
{\'Electrocinétique, Circuits d'ordre 1}{lelay,mines}

On cherche à décrire le circuit suivant :
\begin{circuit}
      \draw
      node [ground] at (0,0) {}
      to [vsource, v^>=$E$, i^>=$i$] (0,2)
      to [short] (1,2)
      
      (1,2) to [short] (1,1)
      to [short] (2,1)
      to [R, l^=$R_2$] (3,1)
      to [short] (4,1)
      to [short] (4,2)
      
      (1,2) to [short] (1,3)
      to [short] (2,3)
      to [R, l^=$R_1$] (3,3)
      to [short] (4,3)
      to [short] (4,2)
      
      (4,2) to [short] (5,2)
      to [R, l^=$R_0$] (6,2)
      to [short] (7,2)
      
      (7,2) to [short] (8,2)
      
      (8,2) to [short] (8,1)
      to [short] (9,1)
      to [L, l^=$L_2$] (10,1)
      to [short] (11,1)
      to [short] (11,2)
      
      (8,2) to [short] (8,3)
      to [short] (9,3)
      to [L, l^=$L_1$] (10,3)
      to [short] (11,3)
      to [short] (11,2)
      
      (11,2) to [short] (12,2)
      to [L, l^=$L_0$] (13,2)
      to [short] (14,2)
      
      to [short] (14,0)
      node [ground]{};
\end{circuit}

\begin{questions}
    \questioncours Loi de composition des résistances en série et en parallèle. (Si vous l'avez vu : ) Pont diviseur de tension.
    \question Écrire l'équation régissant le courant $i$ passant  dans le circuit plus haut et montrez que cette équation peut se mettre sous la forme
    \begin{align*}
        L \dv{i}{t} + Ri = E
    \end{align*}
    Où $L$ (resp. $R$) est une inductance (resp. résistance) fonction de $L_0$, $L_1$ et $L_2$ (resp. $R_0$, $R_1$, $R_2$) que l'on précisera.
    \question La forme revêtue par $L$ vous semble-t-elle familière ? Justifier que les bobines se comportent `un peu comme' des résistances.
    \question Écrire maintenant l'équation régissant l'évolution de la tension $u$ dans le circuit suivant et montrez que cette équation peut se mettre sous la forme
    \begin{align*}
        C \dv{u}{t} + \frac{u}{R} = I
    \end{align*}
    Où $C$ (resp. $R$) est une capacité (resp. résistance) fonction de $C_0$, $C_1$ et $C_2$ (resp. $R_0$, $R_1$, $R_2$) que l'on précisera.
    
\begin{circuit}
      \draw
      node [ground] at (0,0) {}
      to [short] (0,2)
      to [short] (1,2)
      
      (1,2) to [short] (1,1)
      to [short] (2,1)
      to [R, l^=$R_2$] (3,1)
      to [short] (4,1)
      to [short] (4,2)
      
      (1,2) to [short] (1,3)
      to [short] (2,3)
      to [R, l^=$R_1$] (3,3)
      to [short] (4,3)
      to [short] (4,2)
      
      (4,2) to [short] (5,2)
      to [R, l^=$R_0$] (6,2)
      to [short] (7,2)
      
      node [ground] at (7,0) {}
      (7,0) to [isource, v^>=$u$, i^>=$I$] (7,2)
      
      (7,2) to [short] (8,2)
      
      (8,2) to [short] (8,1)
      to [short] (9,1)
      to [C, l^=$C_2$] (10,1)
      to [short] (11,1)
      to [short] (11,2)
      
      (8,2) to [short] (8,3)
      to [short] (9,3)
      to [C, l^=$C_1$] (10,3)
      to [short] (11,3)
      to [short] (11,2)
      
      (11,2) to [short] (12,2)
      to [C, l^=$C_0$] (13,2)
      to [short] (14,2)
      
      to [short] (14,0)
      node [ground]{};
\end{circuit}

    \question Proposer une loi de composition en série et en parallèle des bobines et des condensateurs.
\end{questions}
\end{exercise}


\begin{solution}

\begin{questions}
    \questioncours ok
    \question $E$
    \question $t = 0$, la bobine est un interrupteur ouvert, $t=\infty$, le condensateur est un fil
    \question $3L \dv{u}{t} + R u = 0$
    \question $\tau = 3L/R$
    \question Une exponentielle décroissante pépère.
\end{questions}

\end{solution}