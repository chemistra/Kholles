% Niveau :      PCSI
% Discipline :  Ondes signaux
% Mots clés :   Propagation des ondes

\begin{exercise}{Sono pourrie}{2}{Sup,Spé}
{Ondes,Propagation des ondes,Doppler}{bermudez}

\noindent\textsfbf{Problème ouvert}

Lors d'un concert de Ariana Grande à Bercy, vous avez le malheur de vous trouver au rang ZC à 150 m de la scène. La salle est approximativement de dimensions $\mathrm{150\,m\times 350\,m \times 30\,m}$. Les haut parleurs émettent la musique de part et d'autre de la scène, qui fait 30 mètres de long.

En quoi cette configuration risque d'altérer votre expérience de spectateur ?

\end{exercise}

\begin{solution}

On superpose deux signaux, modélisés comme sinusoidaux mais légèrement déphasés à cause d'une différence de propagation.

Soit la différence entre les deux haut-parleurs. Soit la différence entre le son direct et le son réfléchi sur le plafond. On peut regarder la différence entre les oreilles aussi.

On pourra étudier le temps d'écho également.
\end{solution}