% RC passe bas
\begin{exercise}{Filtres linéaires du premier ordre}{0}{Sup}
{\'Electrocinétique,Filtrage linéaire, Premier ordre}{bermudez}

\begin{minipage}[t]{.3\linewidth}
\vspace{-1.5em}
\begin{circuitikz}
      \draw
      (0,0) to [open, v^=$\underline{e}$, *-o] (0,2)
      (0,2) to [R, l=$R$] (2,2) 
      to [C, l^=$C$, *-*] (2,0)
      (2,2) to [short] (3,2)
      (3,0) to [open, v_=$\underline{s}$, *-o] (3,2)
      (0,0) to [short] (3,0);
\end{circuitikz}
\vspace{1em}
\end{minipage}\begin{minipage}[t]{.7\linewidth}
    \textsf{Question de cours : } Déterminez la nature et le diagramme de Bode de ce filtre, en précisant sa fréquence de coupure et son gain statique.
\end{minipage}
\end{exercise}


% RL passe haut
\begin{exercise}{Filtres linéaires du premier ordre}{0}{Sup}
{\'Electrocinétique,Filtrage linéaire, Premier ordre}{bermudez}

\begin{minipage}[t]{.3\linewidth}
\vspace{-1.5em}
\begin{circuitikz}
      \draw
      (0,0) to [open, v^=$\underline{e}$, *-o] (0,2)
      (0,2) to [R, l=$R$] (2,2) 
      to [L, l^=$L$, *-*] (2,0)
      (2,2) to [short] (3,2)
      (3,0) to [open, v_=$\underline{s}$, *-o] (3,2)
      (0,0) to [short] (3,0);
\end{circuitikz}
\vspace{1em}
\end{minipage}\begin{minipage}[t]{.7\linewidth}
    \textsf{Question de cours : } Déterminez la nature et le diagramme de Bode de ce filtre, en précisant sa fréquence de coupure et son gain statique.
\end{minipage}
\end{exercise}


% RC passe bas
\begin{exercise}{Filtres linéaires du premier ordre}{0}{Sup}
{\'Electrocinétique,Filtrage linéaire, Premier ordre}{bermudez}

\begin{minipage}[t]{.3\linewidth}
\vspace{-1.5em}
\begin{circuitikz}
      \draw
      (0,0) to [open, v^=$\underline{e}$, *-o] (0,2)
      (0,2) to [C, l=$C$] (2,2) 
      to [R, l^=$R$, *-*] (2,0)
      (2,2) to [short] (3,2)
      (3,0) to [open, v_=$\underline{s}$, *-o] (3,2)
      (0,0) to [short] (3,0);
\end{circuitikz}
\vspace{1em}
\end{minipage}\begin{minipage}[t]{.7\linewidth}
    \textsf{Question de cours : } Déterminez la nature et le diagramme de Bode de ce filtre, en précisant sa fréquence de coupure et son gain statique.
\end{minipage}
\end{exercise}


% RL passe haut
\begin{exercise}{Filtres linéaires du premier ordre}{0}{Sup}
{\'Electrocinétique,Filtrage linéaire, Premier ordre}{bermudez}

\begin{minipage}[t]{.3\linewidth}
\vspace{-1.5em}
\begin{circuitikz}
      \draw
      (0,0) to [open, v^=$\underline{e}$, *-o] (0,2)
      (0,2) to [L, l=$L$] (2,2) 
      to [R, l^=$R$, *-*] (2,0)
      (2,2) to [short] (3,2)
      (3,0) to [open, v_=$\underline{s}$, *-o] (3,2)
      (0,0) to [short] (3,0);
\end{circuitikz}
\vspace{1em}
\end{minipage}\begin{minipage}[t]{.7\linewidth}
    \textsf{Question de cours : } Déterminez la nature et le diagramme de Bode de ce filtre, en précisant sa fréquence de coupure et son gain statique.
\end{minipage}
\end{exercise}


% Tableau des solutions

{
\def\thesolution{Solutions \arabicAlph{part}\arabicAlph{section}\hspace{2pt}01--\twodigits{exercise}\quad---\quad}
\begin{solution}

\newcounter{subsol}
\newcommand{\parti}{\large\bfseries\sffamily\stepcounter{subsol}\arabicAlph{part}\arabicAlph{section}\hspace{2pt}\twodigits{subsol}\quad}

\noindent\begin{tabularx}{\linewidth}{@{\parti}C@{\vspace{1em}}Ccp{5em}}

\begin{circuitikz}[baseline={(0,2)}]
      \draw
      (0,0) to [open, v^=$\underline{e}$, *-o] (0,2)
      (0,2) to [R, l=$R$] (2,2) 
      to [C, l^=$C$, *-*] (2,0)
      (2,2) to [short] (3,2)
      (3,0) to [open, v_=$\underline{s}$, *-o] (3,2)
      (0,0) to [short] (3,0);
\end{circuitikz} &
$ \underline{H}(\omega) = \dfrac{1}{1+j RC\omega}$ &
Passe-bas &
$H_0 = 1$, \newline $\omega_\text{c} = \dfrac{1}{RC}$ \\

\begin{circuitikz}[baseline={(0,2)}]
      \draw
      (0,0) to [open, v^=$\underline{e}$, *-o] (0,2)
      (0,2) to [R, l=$R$] (2,2) 
      to [L, l^=$L$, *-*] (2,0)
      (2,2) to [short] (3,2)
      (3,0) to [open, v_=$\underline{s}$, *-o] (3,2)
      (0,0) to [short] (3,0);
\end{circuitikz} &
$ \underline{H}(\omega) = \dfrac{j L/R\omega}{1+j L/R\omega}$ &
Passe-haut &
$H_0 = 1$, \newline $\omega_\text{c} = \dfrac{R}{L}$ \\

\begin{circuitikz}[baseline={(0,2)}]
      \draw
      (0,0) to [open, v^=$\underline{e}$, *-o] (0,2)
      (0,2) to [C, l=$C$] (2,2) 
      to [R, l^=$R$, *-*] (2,0)
      (2,2) to [short] (3,2)
      (3,0) to [open, v_=$\underline{s}$, *-o] (3,2)
      (0,0) to [short] (3,0);
\end{circuitikz} &
$ \underline{H}(\omega) = \dfrac{jRC\omega}{1+j RC\omega}$ &
Passe-haut &
$H_0 = 1$, \newline $\omega_\text{c} = \dfrac{1}{RC}$ \\

\begin{circuitikz}[baseline={(0,2)}]
      \draw
      (0,0) to [open, v^=$\underline{e}$, *-o] (0,2)
      (0,2) to [L, l=$L$] (2,2) 
      to [R, l^=$R$, *-*] (2,0)
      (2,2) to [short] (3,2)
      (3,0) to [open, v_=$\underline{s}$, *-o] (3,2)
      (0,0) to [short] (3,0);
\end{circuitikz} &
$ \underline{H}(\omega) = \dfrac{1}{1+j L/R\omega}$ &
Passe-bas &
$H_0 = 1$, \newline $\omega_\text{c} = \dfrac{R}{L}$ \\

\end{tabularx}

\end{solution}

}
