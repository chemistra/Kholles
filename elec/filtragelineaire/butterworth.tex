% Niveau :      PCSI - PC
% Discipline :  Elec
% Mots clés :   Elec, Ordre 3

\begin{exercise}{Filtre de Butterworth}{1}{Sup,Spé}
{\'Electrocinétique}{lelay}

On donne le filtre de Butterworth d'ordre trois :

\begin{circuit}
      \draw (0,0) 
      to [short, o-] (1,0)
      to [L, l=$L_1$] (3,0)
      to [L, l=$L_2$] (5,0)
      to [short, -*] (6,0);
      
      \draw (0,-2) 
      to [short, o-*] (6,-2);
      
      \draw (3,0) 
      to [C, l=$C$] (3,-2);
      
      \draw (5,0) 
      to [R, l=$R$] (5,-2);
\end{circuit}

tel que $L_1/L_2 = 8$ et $R^2 C/L_2 = 27/8$

\begin{questions}
    \questioncours Pour chaque composant du circuit, donner la loi associée et le comportement asymptotique.
    \question Qualitativement, donner le comportement asymptotique de ce filtre. Pourquoi selon-vous est-il appelé d'ordre trois ?
    \question Déterminer la fonction de transfert du filtre et tracer son gain en fonction de la fréquence. Donner la pulsation de coupure en fonction de $L_2$
    \question Quel est l'intérêt d'utiliser ce type de filtre ? On comparera aux filtres similaires vus en classes préparatoires.
\end{questions}
\end{exercise}

\begin{solution}
    % $$H(\omega) = \dfrac{1}{1 + j Q \qty(\dfrac{\omega}{\omega_0} - \dfrac{\omega_0}{\omega})}, Q=R\sqrt{\dfrac{C}{L}}, \omega_0 = \dfrac{1}{\sqrt{LC}}$$
    
    \begin{align*}
        H(\omega) &= \frac{1}{1 +\frac{L_1+L_2}{R} j \omega  + L_1 C (j\omega)^2 +\frac{L_1L_2C}{R}(j\omega)^3} \\ 
        &= \frac{1}{1 + 3\qty(\frac{3L_2}{R}j\omega) + 3 \qty(\frac{3L_2}{R}j\omega)^2 + \qty(\frac{3L_2}{R}j\omega)^3} \qqtext{d'apres l'enonce}\\
        &= \frac{1}{\qty(1 + j\frac{\omega}{\omega_0})^3} \qqtext{avec} \omega_0  = \qty(\frac{R}{L_1 L_2 C})^{1/3} = \frac{3L_2}{R}
    \end{align*}
    
    $$ G(\omega) = \frac{1}{\sqrt{1 + \qty(\frac{\omega}{\omega_0})^6}}$$
    
    % $$ \omega_0  = \qty(\frac{R}{L_1 L_2 C})^{1/3}$$
    
    Ce filtre est un passe bas qui présente une meilleure sélectivité que ceux étudiés en prépa (-60 dB / décade) et qui n'introduit pas de résonance (comme le filtre RLC d'ordre 2) qui pourrait déformer le signal. 
\end{solution}