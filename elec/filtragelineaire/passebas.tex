% Niveau :      PCSI - PC
% Discipline :  Elec
% Mots clés :   Elec, Ordre 2

\begin{exercise}{Construction d'un passe bas}{1}{Sup,Spé}
{\'Electrocinétique, Circuits d'ordre 2}{lelay}

\begin{questions}
    \questioncours Présentez brièvement les différents types de filtres au programme
    \question À l'aide des composants électroniques usuels, construire un passe-bas d'ordre 2 simple. Vérifier son comportement par une analyse asymptotique.
    \question Donner la fonction de transfert de ce filtre. Tracer le diagramme de Bode asymptotique associé. 
    \question Comment choisir R, L et C pour avoir une fréquence de coupure de 1 kHz et une résonnance d'amplitude 20 dB ? 
\end{questions}
\end{exercise}

\begin{solution}
    \begin{circuit}
      \draw
      (0,0) to [short] (5,0)
      to [open, v_=$s$, *-o] (5,2) 
      to [short] (4,2)
      (0,0) to [open, v^=$e$,*-o] (0,2)
      to [R, l=$R$] (2,2)
      to [L, l=$L$, -*] (4,2) 
      to [C, l^=$C$] (4,0);
    \end{circuit}
    
    Ensuite, $\omega_0 = \frac1{\sqrt{LC}} = 10$ et $G_{dB} = 20\log{H_{max}} = 20$ donc $H_{max} \simeq Q = \frac1R \sqrt{\frac{L}{C}}= 10$
\end{solution}