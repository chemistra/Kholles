% Niveau :      PC
% Discipline :  Electromagnétisme
%Mots clés :    Equations de Maxwell, Drude

\begin{exercise}{Modèle de Drude pour les isolants}{2}{Spé}
{Ondes électromagnétiques,Plasma,Diélectrique}{bermu}

\paragraph{Rappel :} le modèle de Drude décrit la dynamique des électrons dans un champ électrique $\vE$ de la manière suivante
$$m\dv[2]{\vv}{t} = q\vE - m\gamma\pdv{\vr}{t}.$$
où $\vr$ est la position de la particule, $m$ sa masse, $q$ sa charge.
\begin{questions}
    \questioncours Rappeler les conditions du modèle de Drude et la relation de dispersion des ondes dans un conducteur décrit par ce modèle.\\
    On introduira et estimera un ordre de grandeur de la fréquence $\gamma$, la densité du plasma $n$, la pulsation plasma $\omega_\textsc{p}$ et de la conductivité $\sigma$.
\begin{EnvUplevel}
    Afin de généraliser ce modèle pour un isolant, Drude a également proposé le modèle suivant
    $$m\dv[2]{\vv}{t} = q\vE -m{\omega_0}^2\vr - m\gamma\pdv{\vr}{t}.$$
\end{EnvUplevel}
    \question Rappeler ce qu'est un diélectrique. Quelle différence y a-t-il avec un conducteur ? En déduire le sens du terme en $\omega_0$. \\
    Pour un diélectrique $\omega_0$ est dans le domaine visible.
    \question Déduire du modèle précédent la polarisabilité $\alpha$ du diélectrique, telle que le moment dipolaire
    $$\vp = \alpha\vE.$$
    Quelle est l'unité de $\alpha$ ?
    \question En utilisant le fait que la perméabilité $\varepsilon(\omega)$ d'un diélectrique s'écrit
    $$\varepsilon(\omega) = 1 + n\alpha = \dfrac{k^2c^2}{\omega^2},$$
    déduire de la question précédente la relation de dispersion des électrons.
    \question Montrer que pour $\gamma$ faible (on dira devant quoi et on vérifiera si cela est vrai) que la relation de dispersion des ondes peut s'écrire
    $$k^2 = \dfrac{\omega^2}{c^2}\:\dfrac{\omega^2 - {\omega_\infty}^2}{\omega^2 - {\omega_0}^2},$$
    avec $\omega_\infty$ dont on donnera l'expression et le sens physique.
    \question Tracer la relation de dispersion et montrer qu'il existe une bande interdite en fréquence et comparer au cas du simple conducteur.
    \question Peut-on construire une cage de Faraday avec une bâche en plastique ?
\end{questions}

\paragraph{Données :}
\begin{itemize}
    \item charge élémentaire $e = 1,602\times 10^{-19}$ C,
    \item masse le l'électron $m_e = 9,109\times 10^{-31}$ kg,
    \item vitesse de la lumière dans le vide $c = 2,998 \times 10^8$ m$\cdot$s$^{-1}$,
    \item permittivité du vide $\varepsilon_0 = 8,854\times 10^{-12}$ F$\cdot$m$^{-1}$,
\end{itemize}
\end{exercise}