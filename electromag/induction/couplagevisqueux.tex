\begin{exercise}{Couplage visqueux}{2}{Sup}
{Induction}{lelay}

On considère deux rails conducteurs parallèles écartés d'une distance $a$. Perpendiculairement à ces rails on place deux barres, chacune de résistance $R/2$. Les barres sont chacune reliées à un bâti isolant par un ressort. On plonge le tout dans un champ constant $B_0$ orthogonal à la fois aux rails et aux barres.

\begin{questions}
    \questioncours Force de Laplace
    \question Exprimer $\Phi$ le flux du champ magnétique entre les deux barres en fonction de $B_0$, $a$ et les positions $x_1$ et $x_2$ des deux barres.
    \question En déduire une expression de l'intensité $I(t)$ parcourant le circuit
    \question Exprimer les forces de Laplace s'appliquant sur les barres et en déduire l'équation de la dynamique pour chacune des deux barres.
    \question Initialement chaque barre est fixée à la position de repos du ressort auquel elle est attaché. Une des barres est lancée avec une vitesse $v_0$. Décrire l'évolution du système au long terme.
\end{questions}

\end{exercise}

\begin{solution}
    \begin{questions}
        \questioncours $\dd{\vec{F}} = I\dd{\vec{\ell}}\cross\vec{B}$
        \question $\Phi = a B(x_2 - x_1 + \ell_0)$
        \question Resistance totale $R$. $e = RI = \dv{\Phi}{t} = \dfrac{B_0 a}{R}(\dot{x}_2-\dot{x}_1)$
        \question $F_1 = -IaB_0$, $F_2 = IaB_0$, d'où :
        \begin{align*}
            m\ddot{x}_1 &= -kx_1-\gamma(\dot{x}_1-\dot{x}_2) &
            m\ddot{x}_2 &= -kx_2+\gamma(\dot{x}_1-\dot{x}_2) &
            \gamma &= \dfrac{(B_0 a)^2}{R}
        \end{align*}
        \question On pose le changement de variable : $\sigma = x_1 + x_2$, $\delta = x_1 - x_2$.
        \begin{align*}
            m\ddot{\sigma} &= -k\sigma &
            m\ddot{\delta} &= -k\delta-\gamma\dot{\delta}
        \end{align*}
        d'où
        \begin{align*}
            \sigma &= \dfrac{v_0}{\omega}\sin\omega_0 t &
            \delta &= \dfrac{v_0}{\omega}\sin\omega_1 t e^{-t/\tau_1} &
            \omega_0 &= \sqrt{\dfrac{k}{m}} & \omega_1 + i\tau_1 &= \dfrac{1}{2}\qty(\omega_0 \pm \sqrt{(\omega_0)^2 - 4\dfrac{\gamma}{m}})
        \end{align*}
    \end{questions}
\end{solution}

