% Niveau :      PC
% Discipline :  Electromagnétisme
%Mots clés :    Equations de Maxwell, Debye-Hückel

\begin{exercise}{Forces de Van der Waals}{3}{Spé}
{\'Electromagnétisme,\'Electrostatique,Statique des fluides}{bermu}

\begin{questions}
    \questioncours Quelle est l'expression du potentiel $V(\vr)$ et du champ $\vE(\vr)$ électriques de deux charges ponctuelles de signes opposés $\pm q$ distantes de $d$ ?
    \question Obtenir l'expression de la force et de l'énergie potentielle $\En_p$ exercée sur ce dipôle par un champ électrique extérieur $\vE_\text{ext}$.
    \question Rappeler le principe des forces de Van der Waals.
   
\begin{EnvUplevel}

On se place en deux dimensions. On considère deux molécules modélisées comme des dipôles permanents $\vp_1$ et $\vp_2$ fixées en $x = \pm d/2, y = 0$. On utilisera la notation
$$\kappa = \dfrac{p_1 p_2}{4\pi\varepsilon_0}.$$
\end{EnvUplevel}
    \question Donner l'énergie potentielle d'interaction des deux dipôles.
    \question \'Etablir les équations de la dynamique de $\vp_2$, $\vp_1$ étant constant.
    \question Donner la force appliquée à $\vp_1$ due à $\vp_2$.
    \question \`A l'aide des deux questions précédentes, donner la force moyenne subie par $\vp_1$ considéré comme fixe sachant que $\vp_2$ varie, étant soumis à la dynamique calculée à la question précédente.
\end{questions}


\end{exercise}
