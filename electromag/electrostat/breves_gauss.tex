\begin{exercise}{Condensateur plan}{1}{Spé}
{Electromagnetisme, Theoreme de gauss}{lelay,bermudez}

\paragraph{Point méthode en électromagnétisme :}
\begin{itemize}
    \item à partir de la géométrie de problème, choisir le bon système de coordonnées ;
    \item utiliser les invariances du problème pour réduire le nombre de variables ;
    \item utiliser les symétries pour réduire le nombre de composantes ;
    \item identifier une surface fermée, un contour \emph{etc.} et appliquer le théorème intégral pertinent.
\end{itemize}

\begin{questions}
    \questioncours Énoncé du théorème de Gauss. \\
    Donner sa version locale. Comment passe-t-on de l'un à l'autre ?
    \uplevel{On considère un plan infini dans le vide, en $z=0$, uniformément chargé par une charge surfacique~$\sigma$.}
    \question Donner l'expression du champ électrique $\vec{E}$ engendré par le plan chargé, en veillant bien à distinguer plusieurs cas.
    \uplevel{On considère maintenant deux plans infinis et parallèles, l'un de charge surfacique $\sigma$ placé en $z = -d/2$ et l'autre de charge surfacique $-\sigma$ placé en $z = d/2$.}
    \question Quel est le champ électrique résultant hors des deux plans ($\abs{z} > d/2$) ? Quel est le champ électrique entre les deux plans ($\abs{z} < d/2$) ?
    \question Quel est le potentiel électrique $V(z)$ à l'intérieur et à l'extérieur du condensateur.
    \question On appelle $U$ la différence de potentiel entre les deux plans. Exprimer $U$ en fonction de $\sigma$, $d$ et $\epsilon_0$.
    \question Dans un condensateur électrique, quelle est le lien entre sa charge $q$, la différence de potentiel $U$ sa capacité $C$ ? En déduire le lien entre $C$ et les caractéristiques géométriques du condensateur (sa surface $S$, nottament). 
    \uplevel{En réalité, le milieu entre les deux armatures n'est pas le vide mais un diéléctrique (il faut donc remplacer $\ep_0$ par $\ep_0\ep_r$).}
    \question \textsf{Application numérique :} condensateur céramique constitué d'armatures circulaires de diamètre 5 mm et d'une épaisseur de 1 $\mu$m de céramique de permittivité $\ep_r = 200$, $\ep_0 = \SI{8.9e-12}{F.m^{-1}}$.
								  
    %\questionbonus On suppose maintenant que l'espace entre les plaques la distribution suivante de charge volumique dans l'espace : $\rho(x, y, z \geq 0) = \rho_0 e^{-z/\lambda}$ où $\lambda$ est une constante, et $\rho(x, y, z<0) = 0$. Quel est le champ électrique $\vec{E}(x, y, z)$ pour $z < 0$ ? Pour $z \geq 0$ ?
\end{questions}

\end{exercise}


\begin{exercise}{Cylindre infini}{1}{Spé}
{Electromagnetisme, Theoreme de gauss}{lelay,bermudez}

\paragraph{Point méthode en électromagnétisme :}
\begin{itemize}
    \item à partir de la géométrie de problème, choisir le bon système de coordonnées ;
    \item utiliser les invariances du problème pour réduire le nombre de variables ;
    \item utiliser les symétries pour réduire le nombre de composantes ;
    \item identifier une surface fermée, un contour \emph{etc.} et appliquer le théorème intégral pertinent.
\end{itemize}

\begin{questions}
    \questioncours Énoncé du théorème de Gauss. \\
    Donner sa version locale. Comment passe-t-on de l'un à l'autre ?
    \uplevel{On considère un cylindre de rayon $R$ de longueur infinie dans le vide, uniformément chargé par une charge volumique $\rho$.}
    \question Donner l'expression du champ électrique $\vec{E}$ engendré par le cylindre chargé, en veillant bien à distinguer plusieurs cas.
    \question Donner l'expression du potentiel électrique $V(r)$ correspondant. Que se passe-t-il quand $r$ tend vers l'infini ? Est-ce physique ? Quelle hypothèse de départ cause ce problème ?
    %\question On considère maintenant la quantité $u = \frac12 \epsilon_0 \vec{E}^2$, qui a la dimension d'une énergie par unité de volume. Quelle est l'énergie contenue dans un cylindre de rayon $r \leq R$ et de hauteur $H$ ? Même question pour $r \geq R$.
								  
    %\question En remplaçant le cylindre infini par une sphère de rayon $R$, calculez l'énergie contenue dans une sphère de rayon $r \geq R$. Quelle est la limite de cette énergie lorsque $r$ tend vers l'infini ?
\end{questions}

\end{exercise}



\begin{exercise}{La Terre creuse}{1}{Spé}
{Electromagnetisme, Theoreme de gauss}{lelay,bermudez}

\paragraph{Point méthode en électromagnétisme :}
\begin{itemize}
    \item à partir de la géométrie de problème, choisir le bon système de coordonnées ;
    \item utiliser les invariances du problème pour réduire le nombre de variables ;
    \item utiliser les symétries pour réduire le nombre de composantes ;
    \item identifier une surface fermée, un contour \emph{etc.} et appliquer le théorème intégral pertinent.
\end{itemize}

\begin{questions}
    \questioncours Énoncé du théorème de Gauss. \\
    Donner sa version locale. Comment passe-t-on de l'un à l'autre ?
    \uplevel{On considère une sphère de rayon $R_1$ dans le vide, uniformément chargée par une charge volumique $\rho$.}
    \question Donner l'expression du champ électrique $\vec{E}_1$  engendré par la sphère chargée, en veillant bien à distinguer plusieurs cas.
    \question On considère maintenant une autre sphère dans le vide de charge $-\rho$ et de rayon $R_2 < R_1$. Quel est le champ $\vec{E}_2$ généré par cette seconde sphère ?
    \question En déduire le champ engendré une sphère creuse, uniformément chargée par une densité volumique de charge $\rho$ comprise entre les rayons interne $R_2$ et externe $R_1$.
	 
		  
		  
		  
	  
    \question Que penser des théories de la Terre creuse selon lesquelles la Terre est en fait creuse et qu'une partie de l'humanité pourrait habiter à l'intérieur ?
    %\question On considère maintenant une sphère de rayon $R$ et de charge $Q$ et on introduit la quantité $u = \frac12 \epsilon_0 \vec{E}^2$, qui a la dimension d'une énergie par unité de volume. Quelle est l'énergie contenue dans une sphère de rayon $r \leq R$ ? $r \geq R$ ? Dans tout l'espace ?
\end{questions}

\end{exercise}
