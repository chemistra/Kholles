% Niveau :      PC
% Discipline :  Electromagnétisme
%Mots clés :    Equations de Maxwell, Debye-Hückel

\begin{exercise}{Cerceau -- Dipole}{3}{Spé}
{\'Electromagnétisme,\'Electrostatique,Statique des fluides}{bermu}

\begin{questions}
    \questioncours Quelle est l'expression du potentiel $V(\vr)$ et du champ $\vE(\vr)$ électriques de deux charges ponctuelles de signes opposés $\pm q$ distantes de $d$. \\
    Sans calcul, comment généraliser cela à une distribution continue ?
\begin{EnvUplevel}

On considère une distribution linéaire de charge électrique de forme circulaire, de rayon $R$ et de charge totale $q$. On appelle $O$ le centre du cercle. 

On définit $Oz = (O, \ve_z)$ l'axe de symétrie de la distribution. On étudiera uniquement les effets de la distribution pour une charge test située en $Oz$.
\end{EnvUplevel}
    \question Faire un schéma de la situation et introduire un système de coordonnées pertinent.
    \question Discuter les éléments de symétrie de la distribution et en déduire la forme attendue du champ $\vE(z)$ pour un observateur situé sur $Oz$.
    \question Calculer la contribution au potentiel électrique $\dd{V}(z)$ d'un élément $\dd{q}$ de la distribution --- qu'on exprimera dans le système de coordonnées choisi --- pour un observateur situé en $z$ sur $Oz$.
    \question En déduire le potentiel total $V(z)$ généré par le cercle en $z$, puis le champ électrique $\vE(z)$.
    \question Soit une charge test $q'$ contrainte sur l'axe $Oz$. Etablir la dynamique de $q'$ proche de $z = 0$.
\uplevel{On considère que cette charge $q' = -q$ est désormais fixée en $z = d$.}
    \question Quel est le potentiel total $V'$ ressenti pour un observateur en $z$ ?
    \question Montrez que la distribution totale se comporte comme un dipôle, dont on calculera le moment dipolaire $\vec{p}$.
    \question Quid du cas $d = 0$ ?
\end{questions}


\end{exercise}

\begin{solution}
\begin{questions}
    \questioncours 
    \question
    \question
    \question $\dd{V} = \dfrac{q\dd{\theta}/2\pi}{4\pi\varepsilon_0 \sqrt{R^2 + z^2}}$
    \question $V = \dfrac{q}{4\pi\varepsilon_0 \sqrt{R^2 + z^2}}$, $\vE = \dfrac{q z}{4\pi\varepsilon_0 (R^2 + z^2)^{3/2}}$
\end{questions}
\end{solution}