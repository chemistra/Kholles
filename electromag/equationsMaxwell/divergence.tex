\begin{exercise}{Divergence en cylindrique}{0}{Spé}
{Electromagnetisme, Conservation de la charge}{bermu}

\begin{questions}
    \questioncours Donner les 4 équations de Maxwell avec sources sous forme locale et en déduire l'équation de conservation de la charge. \\
    On notera $\vec{j}(\vr,t)$ et $\rho(\vr,t)$ les densités de courant et de charge.
    
    \uplevel{On considère un tube cylindrique, de hauteur $h$, de rayon $r$ et d'épaisseur $dr$. Ce tube reçoit sur sa surface intérieure un flux de charge uniforme, quantifié par le vecteur densité de courant $\vec{j}(r, \theta, z, t) = j(r, t)\vec{e_r}$, et perd depuis sa surface extérieure un flux de charge, donné par $j(r+dr, t) \vec{e_r}$. On note $Q(t)$ sa charge totale et $\rho(r, t)$ sa densité volumique de charge, supposée uniforme dans le tube.}
    
    \uplevel{On considère une électrode cylindrique de rayon $a$ et de hauteur $H \gg a$ qui émet un courant radial~$I(t)$.}
    
    \question Quelles sont les symétries de $\rho$ et $\vec{j}$ ?
    
    \question On étudie la conservation de la charge dans le cylindre. Effectuer un bilan de charges entre $t$ et $t + \dd{t}$, $r$ et $r + \dd{r}$ et retrouver l'équation de la conservation de la charge précédemment obtenue.

    \question En déduire l'expression de la divergence du champ $\vec{j}$ en coordonnées cylindriques.

\end{questions}

\end{exercise}
