% Niveau :      PC
% Discipline :  Electromagnétisme
%Mots clés :    Equations de Maxwell, Drude

\begin{exercise}{Relaxation des charges dans un métal}{2}{Spé}
{\'Equations de Maxwell, Temps de Maxwell,Plasma résistif}{bermu}

On considère un plasma conducteur constitué d'un gaz neutre d'ions et d'électrons de densité de charges $\rho$ et de conductivité électrique $\sigma$. On considérera les ions immobiles.

\begin{questions}
    \questioncours Établir l'équation de la conservation de la charge à partir des équations de Maxwell.
    \question Montrez que la densité de charges créé un champ électrique $\vE$, induisant lui-même un courant électrique $\vJ$.
    \question \'Etablir l'équation régissant $\rho$. On fera apparaître un temps caractéristique $\tau_\textsc{m}$, le temps de Maxwell, dont on donnera une interprétation physique.
    \question Calculer le temps de Maxwell le cuivre, $\sigma = 6\times 10^8$ $\Omega^{-1}\cdot\text{m}^{-1}$. \`A quelle condition peut-on considérer que les charges dans circuit électrique sont toutes relaxées ?
\end{questions}

\paragraph{Données :}
\begin{itemize}
    \item permittivité du vide $\varepsilon_0 = 8,854\times 10^{-12}$ F$\cdot$m$^{-1}$.
\end{itemize}
\end{exercise}

\begin{solution}
\begin{questions}
    \question $\pdv{\rho}{t} + \div\vJ = 0$
    \question $\div \vE = \dfrac{1}{\sigma}\div\vJ =\dfrac{\rho}{\ep_0}$
    \question $\pdv{\rho}{t} + \dfrac{1}{\tau_\textsc{m}} \rho = 0$ avec $\tau_\textsc{m} = \dfrac{\ep_0}{\sigma}$
    \question $\tau_\textsc{m} = 1 \times 10^{-20}$ s. On doit donc avoir $f < 10^{20}$ Hz : toujours vrai !
\end{questions}
\end{solution}


% Niveau :      PC
% Discipline :  Electromagnétisme
%Mots clés :    Equations de Maxwell, Drude

\begin{exercise}{\'Ecrantage magnétique \textbullet\ Effet Meissner}{2}{Spé}
{\'Equations de Maxwell,Plasma inertiel}{bermu}

On considère un plasma constitué d'un gaz neutre d'ions et d'électrons de densité $n$. On considérera les ions immobiles. On impose dans une région $x<0$ de ce gaz un champ magnétique homogène $\vB_0(t) = B_0(t) \ve_z$. On étudie le régime de relaxation des courants dans $x>0$.

\begin{questions}
    \question Montrez qu'un champ $\vB$ induit un champ $\vE$. On supposera que $\vB = B(t)\ve_z$ et $\vE = E\ve_y$ (le justifier). Donner la relation (1) entre $E$ et $B$.
    \question Montrez que par conséquent, les électrons se mettent en mouvement et établir une équation (2) entre la densité de courant $J$ et le champ $E$.
    \question Montrez que le courant $J$ rétroagit lui-même sur $B$ et donner la relation (3) entre les deux.
    \question En déduire l'équation vérifiée par $E$. On fera apparaître une longueur caractéristique $\ell_\text{p}$, la longueur de London.
    \question Au vu des conditions aux limites, résoudre l'équation pour $B$, puis $E$, puis $J$ et en déduire une interprétation de la longueur de London.
    \question Calculer la longeur de London pour un plasma de densité $n = 10^{23}$ m$^{-3}$.
\end{questions}

\paragraph{Données :}
\begin{itemize}
    \item charge élémentaire $e = 1,602\times 10^{-19}$ C,
    \item masse le l'électron $m_e = 9,109\times 10^{-31}$ kg,
    \item pereabilité du vide $\mu_0 = 1,26 \times 10^{-6}$ $H\cdot$m$^{-1}$.
\end{itemize}
\end{exercise}

\begin{solution}
\begin{questions}
    \question Maxwell Faraday (1) $\pdv{E}{x} = -\pdv{B}{t}$
    \question Newton PFD (2) $\pdv{J}{t} = -\dfrac{ne^2}{m_e} E$
    \question Maxwell Ampère (3) $\pdv{B}{x} = -\mu_0 J$ (on vire le courant de déplacement de l'ordre de $v/c$).
    \question D'où $\pdv[2]{E}{x} = \ell_\text{p}^{-2}E$, $\ell_\text{p} = \dfrac{m_e}{\mu_0 e^2} = \dfrac{c}{\omega_\text{p}}$.
    \question $B(x,t) = B_0(t) e^{-x/\ell_\text{p}}$, $E(x,t) = \ell_\text{p} B'_0(t) e^{-x/\ell_\text{p}}$, $J = J_0 e^{-x/\ell_\text{p}}$.
\end{questions}
\end{solution}






% Niveau :      PC
% Discipline :  Electromagnétisme
%Mots clés :    Equations de Maxwell, Drude

\begin{exercise}{Lampe à plasma}{3}{Spé}
{\'Equations de Maxwell,\'Equation de conservation,Statique des fluides}{bermu}

\begin{questions}
    \questioncours Rappeler l'expression de la conductivité $\sigma$ d'un nuage d'électrons du modèle de Drude en fonction de sa densité $n$ et la fréquence de collision des électrons $\nu_\text{c}$, et d'autres données nécessaires. \\
    Expliciter la relation entre la vitesse $\vv$ des électrons et le potentiel $V(\vr)$ dans lequel ils se trouvent.
\begin{EnvUplevel}
Afin de modéliser une décharge électrique près d'une électrode d'une lampe à plasma, considérons un gaz d'atomes qui s'ionise. On considère alors la densité d'électrons $n_e(\vr)$ et d'ions $n_i(r)$ autour de la boule.

Le nuage électronique créé un champ de pression $p(\vr)$ et un champ électrique $V(\vr)$ à cause des charges.
\end{EnvUplevel}
    \question Considérant que le gaz électronique est à l'équilibre hydrostatique à la température $T$ et que son équation d'état est celle d'un gaz parfait, donner une relation entre la densité du gaz $n_e(\vr)$, le potentiel électrique $V(r)$ et la température $T$. \\
    Cette loi vous rappelle-t-elle quelque chose ? Discuter cette hypothèse.
\uplevel{On modélise la boule plasma comme une électrode quasi plane en $z=0$ portée au potentiel $V_0 = 10$~kV et un gaz d'ions et d'électrons en $z>0$ supposé quasi neutre $n_i = n_e = n$.}
    \question Considérant qu'à chaque collision avec un atome, un électron expulse $\alpha$ électrons et en notant son libre parcours moyen $\ell$ et l'énergie de première ionisation des atomes $V_\text{ion}$ (en volts), estimer par un raisonnement de votre choix $\alpha$.
    \question Justifier que le taux d'ionisation peut s'écrire $\alpha\nu_\text{c}n$ et déduire de la relation de conservation des électrons un lien entre $\vv$ et $n$.
    \question Déduire des questions précédentes l'équation de Schottky
    $$\qty(\grad^2 - {\ell_\textsc{s}}^{-2})n = 0,$$
    en exprimant la longueur de Schottky $\ell_\textsc{s}$ en fonction des données du problème et la vitesse
    $$c_\textsc{s} = \sqrt{\dfrac{k_\textsc{b}T}{m_e}}.$$
    Interpréter physiquement $\ell_\textsc{s}$ et $c_\textsc{s}$.
    \question Quelle est l'équation régissant $V(\vr)$ ?
    \question Résoudre et tracer les profils de potentiel $V(\vr)$, de densité $n(\vr)$ et de vitesse $\vv(\vr)$ ?
    \question Montrez que les particules passent le mur du son. Interpréter. Considérant que la décharge électrique se dissipe à ce moment là, estimer la taille de la décharge électrique.
\end{questions}

\paragraph{Données :}
\begin{itemize}
    \item charge élémentaire $e = 1,602\times 10^{-19}$ C,
    \item masse le l'électron $m_e = 9,109\times 10^{-31}$ kg,
    \item permittivité du vide $\varepsilon_0 = 8,854\times 10^{-12}$ F$\cdot$m$^{-1}$,
    \item vitesse de la lumière dans le vide $c = 2,998 \times 10^8$ m$\cdot$s$^{-1}$,
    \item constante de Boltzmann $k_\textsc{b} = 1,381\times 10^{-23}$ J$\cdot$K$^{-1}$,
    \item Dans une décharge électrique : $T = 10 000$ K, $n = 10^{23}$ m$^{-3}$.
\end{itemize}
\end{exercise}