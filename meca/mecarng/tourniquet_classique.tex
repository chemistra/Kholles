% Niveau :      Spé
% Discipline :  Méca RNG

\begin{exercise}{Bague sur un cerceau centré}{2}{Spé}
{Mécanique,Mécanique en référentiel non galiléen}{lelay, classique}

\begin{questions}
    \questioncours Forces de frottement solides et fluides.
\begin{EnvUplevel}
On considère une bague de masse $m$ enfilée sur un cerceau rigide, qui tourne à vitesse angulaire $\Omega$ autour de son axe de symétrie vertical.
\end{EnvUplevel}
    \question Quel type de frottement intervient entre la bague et le cerceau ? On les négligera par la suite.
    \question Donner l'équation du mouvement de la bague et interpréter qualitativement l'influence de chaque terme sur sa dynamique. On fera intervenir une pulsation caractéristique $\omega_0$ convenablement choisie.
    \question Combien de positions d'équilibre de la bague y a-t-il ? De quoi dépend ce nombre ?
    \question Indiquer la (les) positions d'équilibre(s) de la bague en fonction du paramètre $\mu = \frac{\Omega^2}{\omega_0^2}$. Pour chaque position d'équilibre, donner la condition de stabilité.
    \question Montrer que toutes les forces considérées ici dérivent d'un potentiel et donner la forme de l'énergie potentielle totale $E_p$ de la bague.
    \question Tracer pour plusieurs valeurs pertinentes de $\mu$ le profil d'énergie potentielle et indiquer les positions d'équilibre stables et instables. Identifier les "bassins d'attraction" des points fixes stables.
    \question Pourquoi parle-t-on d'une "bifurcation" ou "transition de phase" du système en $\mu = 1$ ? 
\end{questions}
\end{exercise}