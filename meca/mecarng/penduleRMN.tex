% Niveau :      Spé
% Discipline :  Méca
% Mots clés :   Chaine, caténaire

\begin{exercise}{Analogie RMN et pendule en rotation}{2}{Spé}
{Mécanique,Mécanique en référentiel non galiléen,Pendule}{bermu}

\begin{questions}
    \questioncours Forces d'inertie d'entraînement et de Coriolis.
\begin{EnvUplevel}
Considérons un pendule rigide de longueur $\ell$ et de masse $m$ fixé en $O$ dans un champ de pesanteur $\vg = -g\ve_z$. Le pendule se trouve dans un référentiel $\cal{R}'$ en rotation uniforme à la vitesse $\Omega$ autour de l'axe $\ve_z$. \\
On se placera dans un système de coordonnées sphériques $(r,\theta,\varphi)$ où $\theta\in\qty[0,\pi]$ est l'angle entre $Oz$ et l'axe du pendule.
\end{EnvUplevel}
    \question Chercher les positions d'équilibre du pendule $\theta_\text{eq}$, ainsi que leur stabilité. \\
    On ne fera intervenir que la fréquence propre du pendule $\omega_0$ et la fréquence de rotation $\Omega$.
    \question Tracer pour plusieurs valeurs pertinentes de $\Omega$ le profil d'énergie potentielle ainsi que les positions d'équilibre stables et instables. Interpréter les différentes zones de ce graphe.
    \uplevel{Nous allons maintenant étudier la dynamique locale du pendule proche des positions d'équilibre.}
    \question Pour chaque position d'équilibre $\theta_{\text{eq}, i}$, en substituant $\theta$ dans l'équation de la dynamique du pendule par
    $$\theta = \theta_{\text{eq}, i} + \varepsilon_i,$$
    montrer que pour de faibles variations $\varepsilon_i$ autour de la position d'équilibre, l'équation de la dynamique devient à l'ordre 1 en $\varepsilon_i$
    \begin{itemize}
        \item celle d'un oscillateur harmonique classique de pulsation $\omega_i$, si elle est stable,
        \item celle d'un système exponentiellement divergent avec taux de croissance $\sigma_i$, si elle est instable.
    \end{itemize}
    Interpréter physiquement ces deux résultats en discutant notamment des limites du modèle.
    Tracer $\theta_{\text{eq}, i}$, $\omega_i$ et $\sigma_i$ en fonction de $\Omega$ et interpréter.
\end{questions}
\paragraph{Questions facultatives (niveau PC) :}
\begin{questions}
    \question Rappeler ce qu'est la précession de Larmor et dresser une analogie avec le système précédent.
    \question Rappeler ce qu'est une transition de phase, ou changement d'état. \\
    En quoi peut-on dire que ce système effectue une transition de phase à la pulsation $\omega_c$ ?
\end{questions}

\end{exercise}