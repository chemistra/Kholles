% Niveau :      Spé
% Discipline :  Méca RNG

\begin{exercise}{Bille dans un récipient conique}{2}{Spé}
{Mécanique,Mécanique en référentiel non galiléen}{lelay}

\begin{questions}
    \questioncours Forces conservatives et conservation de l'énergie.
\begin{EnvUplevel}
On considère une bille de masse $m$ libre de rouler sans frottements sur la surface d'un cône d'angle au sommet $\alpha$ renversée dans lequel elle est posée. Ce cône tourne à une fréquence $\Omega$ autour de son axe de révolution, et la bille tourne avec lui de manière synchronisée.
\end{EnvUplevel}
    \question On suppose qu'au départ la bille se trouve à une distance $\ell_0$ de la pointe du cône. Indiquer la condition sur $g$, $\Omega$, $\alpha$ que doit vérifier $\ell_0$ pour que la bille soit à l'équilibre. 
    \question Dire qualitativement si l'équilibre considéré est stable ou instable.
    \question On suppose maintenant que la bille posée sans vitesse initiale à une distance $\ell = \ell_0 - \epsilon$ du sommet. Comment va varier $\ell$ en fonction du signe de $\epsilon$ ?
    \question En considérant $\epsilon \ll 1$, donner le temps que va mettre la bille à retomber au sommet du cône.
    % \question On se place maintenant exactement en $\ell = \ell_0$ et on donne une petite pichenette à la bille afin de lui donner une petite vitesse $\nu$. Comment va varier $\ell$ en fonction du signe de $\nu$ ?
    % \question En considérant $\\nu \ll 1$, donner le temps que va mettre la bille à sortir du cône (on considère que le cône est de taille $L$^). Quelle sera sa vitesse $v_s$ au moment de sortir ?
    % \question Calculer la trajectoire de la bille après sa sortie dans le référentiel tournant puis dans le référentiel du laboratoire
\end{questions}
\end{exercise}