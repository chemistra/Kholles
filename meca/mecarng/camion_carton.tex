% Niveau :      Spé
% Discipline :  Méca
% Mots clés :   Chaine, caténaire

\begin{exercise}{Attention fragile ! (dans le camion)}{2}{Spé}
{Mécanique,Mécanique en référentiel non galiléen,Frottements}{bermu}


\textsf{Question de cours :} Lois du frottement solide de Coulomb.

Lors d'un déménagement, vous transportez vos cartons avec un camion.
Vous craignez qu'une conduite brusque entraîne les cartons à se cogner contre les parois du camion.

\textsf{Question ouverte :} Estimez l'accélération maximale du camion permettant de ne pas casser la vaisselle contenue dans un carton. (\textsl{On pourra s'aider des indications ci-après})

\paragraph{Indications}

\begin{enumerate}
    \item Modéliser la situation de manière simple : un carton de masse $m$, dans un camion de longueur $L$ allant à une accélération constante $\gamma$ et estimer à quelle condition se produit le glissement du carton.
    \item Estimer l'énergie nécessaire à briser la vaisselle dans le carton à partir de considérations simples.
    \item Estimer l'énergie acquise par le carton avant le choc contre la paroi du camion.
    \item Comparer à l'accélération typique d'un véhicule.
\end{enumerate}

On estimera par des ordres de grandeur pertinents les quantités physiques du problème.

\end{exercise}

\begin{solution}
Dans le référentiel en accélération :
    $$m\dv{\vec{v}}{t} = -mg\vec{e}_z + N\vec{e}_z + T\vec{e}_x - m\gamma\vec{e}_x$$
Avec $N = mg$.

Les lois de Coulomb donnent :
\begin{itemize}
    \item en statique : $v = 0$, $T = m\gamma < f_s N = f_s m g$ soit $\gamma < f_s g$ \\
    \noindent il y a donc une première accélération critique qui est $\gamma_\text{c1} = f_s g$.
    \item en dynamique : $T = f_d N$ donc $\ddot{x} = -\gamma + f_d g$
\end{itemize}

Donc on accumule une énergie $E = m L (\gamma - f_d g)$.

On estime que l'énergie nécessaire à fracasser la vaisselle est celle donnée par une chute de $H = 1$~m.

Donc $\gamma_\text{c2} = g\qty(f_sd + \dfrac{H}{L})$.

ODG :


L'angle de glissement limite pour un carton est de l'ordre de 10$^\circ$ donc $f \sim 0.2$ et $\gamma_\text{c1} \sim 0.2g$.

$L = 5$ m. Donc $\gamma_\text{c2} = 0.4$.

$\gamma = 0.3$ g pour un véhicule.

Il faudra donc être prudent.


\end{solution}

% Niveau :      Spé
% Discipline :  Méca
% Mots clés :   Chaine, caténaire

\begin{exercise}{Attention fragile ! (avec la brouette)}{2}{Spé}
{Mécanique,Moment cinétique,Frottements}{bermu}


\textsf{Question de cours :} Lois du frottement solide de Coulomb.

Lors d'un déménagement, vous transportez vos cartons avec une brouette.
Vous craignez que pencher trop la brouette fasse glisser ou se renverser le carton.

\textsf{Question ouverte :} Estimez les angles à partir desquels le carton se renverse ou glisse.

\paragraph{Indications}~\\

On modélisere la situation de manière simple : un carton de masse $m$, de hauteur $H$ et de longueur de largeur identiques $L$, posé sur un plan incliné d'un angle $\theta$ en étudiant :
\begin{enumerate}
    \item Etudier la situation du glissement.
    \item Etudier la situation du versage.
\end{enumerate}

On estimera par des ordres de grandeur pertinents les quantités physiques du problème.

\end{exercise}

\begin{solution}
On a dans le référentiel penché :
    $$m\dv{\vec{v}}{t} = m\vec{g} + N\vec{e}_y + T\vec{e}_x$$
Avec $\vec{g} = -g(\cos\theta\vec{e}_y + \sin\theta\vec{e}_x)$

Les lois de Coulomb donnent :
\begin{itemize}
    \item en statique : $v = 0$, $T = m g \sin\theta < f_s N = f_s m g \cos\theta$ soit $\tan\theta < f_s$ \\
    \noindent il y a donc l'angle de glisement qui est $\theta_\text{g} = \tan^{-1}f_s$.
\end{itemize}

Pour le glissement, on applique le TMC au point inférieur du carton $A$.

$$J = \dfrac{m}{HL^2} \iiint \dd{x}\dd{y}\dd{z} (x^2+y^2) = \dfrac{1}{3}m(L^2+H^2)$$

$$\scr{M}^{Oz}_{T} = \scr{M}^{Oz}_{R} = \vec{0}$$

$$\scr{M}^{Oz}_{m\vec{g}} = \dfrac{m}{HL^2z} \iiint \dd{x}\dd{y}\dd{z} \vec{r}\cross\vg = m\vec{\text{AG}}\cross\vec{g} = \dfrac{mg}{2}(L \cos\theta - H\sin\theta) =  -\dfrac{mg}{2}\sqrt{L^2+H^2}\sin(\theta-\theta_\text{b}),$$
avec G le barycentre du système et $\theta_\text{b} = \tan^{-1}\dfrac{L}{H}$.

Le système basculera donc si on passe l'angle de basculement $\theta_\text{b}$
\end{solution}