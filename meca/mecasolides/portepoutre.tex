\begin{exercise}{La poutre}{2}{Sup}
{Mécanique,Moment cinétique, Couple}{lelay}

On considère une poutre de longueur $L$ et de masse $M$, qui est portée à l'horizontale par deux bûcherons. L'un d'eux se trouve à l'une des extrémités de la poutre, l'autre se trouve à une distance $\ell$ du premier. Les bûcherons ne peuvent exercer sur la poutre que des forces dirigées verticalement.

\begin{questions}
    \questioncours Moment d'une force en un point.
    \question Sans calculs, quel est la force exercée par chacun des bûcherons sur la poutre dans le cas (a) $\ell = L$ et $\ell = L/2$ ?
    \question Montrer que l'équilibre de la poutre en position horizontale se traduit par un système de deux équations.
    \question Tracer la force exercée par chaque bûcheron pour la maintenir position horizontale en fonction de $\ell$. Que se passe-t-il pour $\ell \rightarrow 0$ ?
    \question Le bûcheron situé à l'extrémité de la poutre se suspend à elle pour faire une farce à son collègue. Quel est le poids que celui-çi doit alors soulever ? Où doit-il se placer sur la poutre ?
\end{questions}

\end{exercise}

\begin{solution}
    \begin{questions}
    \questioncours $M = OP x F$
    \question $\ell = L$ : chacun soulève un poids $M/2$. $\ell = L/2$ : le bucheron au centre soulève toute la poutre, l'autre ne sert à rien.
    \question Équilibre des forces (PFD sur la poutre) : $F_1 + F_2 = Mg$. Équilibre des moments (TMC sur la poutre) $F_1 \frac{L}{2} + F_2 \qty(\frac{L}{2}-\ell)$
    \question $F_2 = Mg\frac{L}{2\ell}$ ; $F_1 = Mg\qty(1 - \frac{L}{2\ell})$
    \question En notant $m$ la masse du bûcheron farceur, $F_1 = -mg$ et donc $F_2 = (M+m)g = Mg\frac{L}{2\ell}$ d'où $\ell = \frac{L}{2}\frac{M}{M+m}$. On a bien $\ell < L/2$ puisque $F_1 < 0$
    \end{questions}
\end{solution}