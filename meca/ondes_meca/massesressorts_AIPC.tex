\begin{exercise}{Propagation dans une chaîne de masses-ressorts}{2}{Spé}
{Propagation}{lelay, mines, X}

On s'intéresse à une chaîne infinie de masses $m$ écartées d'une distance $a$ reliées deux à deux par des ressorts de raideur $k$ et de longueur à vide $a$. On note $X_n$ le déplacement de la $n$-ième masse par rapport à sa position d'équilibre.

\begin{center}
\begin{tikzpicture}

\newcommand\heighta{.75}
\newcommand\heightarrow{.5}

\draw[<->] (-2,\heighta) -- (0,\heighta);
\draw (-1,1) node[above=2pt] {$a$};

\draw[<->] (0,\heighta) -- (2,\heighta);
\draw (1,1) node[above=2pt] {$a$};

\node[circle,fill=black] (xn) at (0+0.4,0) {};
\draw[thick, dotted] (0,\heightarrow) -- (0,-\heightarrow);
\draw[->] (0,-\heighta) -- (0+0.4,-\heighta);
\draw (0+0.2,-\heighta) node[below=2pt] {$x_{n}$};

\node[circle,fill=black] (xnp) at (2+.8,0) {};
\draw[thick, dotted] (2,\heightarrow) -- (2,-\heightarrow);
\draw[->] (2,-\heighta) -- (2+.8,-\heighta);
\draw (2+.4,-\heighta) node[below=2pt] {$x_{n+1}$};

\node[circle,fill=black] (xnm) at (-2+0.6,0) {};
\draw[thick, dotted] (-2,\heightarrow) -- (-2,-\heightarrow);
\draw[->] (-2,-\heighta) -- (-2+.6,-\heighta);
\draw (-2+.3,-\heighta) node[below=2pt] {$x_{n-1}$};


\node[circle,fill=black] (xnmm) at (-3.6,0) {};
\node[circle,fill=black] (xnpp) at (4,0) {};

\draw[decoration={aspect=0.3, segment length=1.5mm, amplitude=2mm,coil},decorate] (xn) -- (xnp);
\draw[decoration={aspect=0.3, segment length=1.2mm, amplitude=2mm,coil},decorate] (xnm) -- (xn);
\draw[decoration={aspect=0.3, segment length=1.8mm, amplitude=2mm,coil},decorate] (xnmm) -- (xnm);
\draw[decoration={aspect=0.3, segment length=1.5mm, amplitude=2mm,coil},decorate] (xnp) -- (xnpp);
\fill[white] (-4.5,1) rectangle (-3.5,-1);
\fill[white] (3.5,1) rectangle (4.5,-1);

\path (xn) -- coordinate[midway](m) (xnp);
\draw (m) node[above=6pt] {$k$};

\path (xn) -- coordinate[midway](n) (xnm);
\draw (n) node[above=6pt] {$k$};

\end{tikzpicture}
\end{center}

\begin{questions}
    \question Établir l'équation du mouvement de la $n$-ième masse en supposant que les déplacements des masses sont faibles par rapport à $a$.
    \question On définit $X$ tel que $X_n(t) = X(x = na, t)$. En supposant $a$ petit devant la taille typique des variations spatiales de $X$, établir l'équation différentielle aux dérivées partielles vérifiée par $X$.
    \question Donne la célérité $c$ des ondes de déplacement se propageant dans la chaîne. Que peut-on dire de cette propagation ?
    \question Comment interpréter les quantités $m_l = m/a$ et $\kappa = k\ a$ ?
    \question Interpréter la quantité $Z = \sqrt{\dfrac{m_l}{1/\kappa}}$
\end{questions}


On cherche maintenant à modéliser un matériau dense (un métal, par exemple) par un réseau cubique, d'arête $a$, avec aux sommets des atomes de masse $m$. On modélise la cohésion du cristal par des ressorts de raideur $k$ et de longueur à vide $a$ présents entre chaque couple d'atomes voisins. On note $(x, y, z)$ les trois directions principales de ce réseau et on considère que les atomes placés dans un même plan $(y, z)$ restent toujours coplanaires (déformation purement longitudinale). L'aire macroscopique d'un de ces plans est notée $S$.

On s'intéresse à la déformation du matériau suivante : tous les atomes des plans initialement situés à des abscisses $x > x_0$ sont décalés d'une distance $\delta x$.

\begin{questions}
    \question Exprimer la contrainte exercée dans le matériau en fonction de $S$, $k$, $a$ et $\delta x$
    \question Exprimer différemment cette contrainte en utilisdant cette fois une grandeur macroscopique, le module d'Young $E$ du matériau.
    
    \question En déduire une expression du module d'Young en fonction de $k$ et $a$. Estimer numériquement sa valeur en ordres de grandeur.
    
    \question Exprimer la vitesse des ondes sismiques longitudinales dans le matériau en fonction du module d'Young et d'une autre quantité qu'on interprétera. Estimer numériquement sa valeur en ordres de grandeur.

\end{questions}

\end{exercise}