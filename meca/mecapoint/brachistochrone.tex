% Niveau :      PC **
% Discipline :  Méca
% Mots clés :   Brachistochrone, moindre action

\begin{exercise}{Brachistochrone}{4}{Spé}
{Mécanique,Mécanique du point}{bermu}

La brachistochrone (de \emph{chronos} le temps, \emph{brakhistos} le plus court) est la trajectoire la plus rapide (en temps) d’un point $A$ à un point $B$.
\begin{questions}
\question Décrire plusieurs situations où la brachistochrone n’est pas la droite $AB$, c'est à dire où le chemin le plus court n'est pas le plus rapide. \\
\textbf{Indice :} On pourra observer une masse $m$ dans un champ de pesanteur $g$ ayant pour trajectoires
\begin{parts}
\part la droite $AB$, \part un arc de cercle, \part une chute verticale un mouvement horizontal...
\end{parts}
\question Analogie avec l'optique.
\begin{parts}
\part Énoncer un principe optique analogue à celui de la brachistochrone.
\part \`A l’interface de deux milieux optiques, quelle loi caractérise ce principe ? Démontrer cette loi.

Supposons maintenant que le milieu soit à gradient d’indice optique $n(z)$.
\part En utilisant la loi précédente, exprimer en fonction de $\theta$, l’angle de la tangente à la trajectoire par rapport à l’axe $\hat{u}_z$, et de $n(z)$ une quantité qui se conserve sur le rayon lumineux.\\
Tracer qualitativement le rayon dans un cas simple.
\end{parts}
\question Retour sur la brachistochrone :
On considère une masse m dans un champ de pesanteur g
\begin{parts}
\part Donner une relation entre la vitesse v, et l’altitude z de m.
\part Par analogie avec la question 2c), donner une relation entre l’angle $\theta$ et l’altitude $z$.
\part Montrer qu’une cycloïde vérifie une telle relation et déduire l’équation de la cycloïde. Tracer cette trajectoire et la comparer aux trajectoires étudiées en 1. \\
\textbf{Donnée :} une cycloïde est la trajectoire d’un point fixé à un cercle qui roule sans glisser sur une droite qui
peut être paramétrée de la manière suivante :
$$\left\{\mqty{x(\nu) = R(\nu - \cos\nu) \\ z(\nu) = R(1 - \sin\nu)}\right., \nu\in[0,\pi].$$
\end{parts}
\end{questions}

\plusloin
On peut également montrer que la brachistochrone est aussi une tautochrone (de \emph{chronos} le temps, \emph{tautos} ) c’est-à-dire une courbe telle que tout point matériel lâché sans vitesse initiale sur la courbe arrive en un point donné en un temps indépendant du point de départ.

\end{exercise}