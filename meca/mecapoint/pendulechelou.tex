% Niveau :      PCSI
% Discipline :  Méca
% Mots clés :   Pendule

\begin{exercise}{Pendule compensé}{2}{Sup, Spé}
{Mécanique,Mécanique du point,Pendule}{lelay}

\begin{questions}
    \questioncours Donner l'équation du mouvement d'un pendule sans frottements
    \question On s'intéresse à un pendule de masse $m$ attaché à une corde de longueur $2L$ reposant sur 2 poulies dont l'autre extrémité est attachée à une masse $M$. La masse $M$ ne peut que monter et descendre. Initialement, la longueur du pendule est $L$.
    \begin{parts}
        \part Discuter de l'évolution du système.
        \part Donner les équations qui régissent le problème.
        \part On considère $M=m$ et on se place dans l'approximation des petits angles. Que se passe-t-il si à l'origine les deux masses sont immobiles et le pendule un peu désaxé ?
        \part Comment alors obtenir un comportement périodique ?
    \end{parts}
\end{questions}
\end{exercise}