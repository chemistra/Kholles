% Niveau :      PCSI
% Discipline :  Méca

\begin{exercise}{Oscillateur anharmonique}{2}{Sup}
{Mécanique,Ressort}{bermudez}

\begin{questions}
    \questioncours Oscillateurs en mécanique : profil d'énergie potentielle et période. On donnera deux exemples.
    
    \uplevel{On considère une masse $m$ liée à deux ressorts de raideur $k$ et de longueur à vide $\ell_0$ dont les points d'attache sont situés de part et d'autre de la masse, sur l'axe $Ox$, en $x = \pm a$. La masse est libre de se déplacer verticalement suivant $z$. On négligera la gravité.}
    \question Montrez qualitativement que si la masse est initialement centrée en $x=0$, elle y restera tout le temps. On considérera cette condition comme vérifiée par la suite.
    \question Dériver l'énergie potentielle $U(z)$ de la masse, en fonction des données du problème. Puis, écrire l'équation de la dynamique du système en fonction de $m$, $z(t)$ et $U(z)$, et leurs dérivées.
    \question Rappeler, dans le cadre des oscillateurs mécaniques, l'hypothèse usuelle faite sur l'amplitude des oscillations du système et appliquer cette hypothèse pour simplifier l'expression de $U(z)$. \\Au regard de la \textsfbf{Q\,1}, interpréter cette expression.
    \question En supposant $a > \ell_0$, donner l'équation de la dynamique de $m$ en faisant apparaître une pulsation caractéristique $\omega$, dont on donnera l'expression.
    \question On suppose maintenant $a = \ell_0$. L'approximation faire à la \textsfbf{Q\,4} est-elle encore valide ? Développer $U(z)$ à l'ordre supérieur en $z/a$.
    \question Donner la nouvelle équation de la dynamique du système. Interpréter la qualification d'oscillateur anharmonique.
    \question Effectuer une séparation des variables $z$ et $t$ de sorte à avoir une équation sur $\dot{z}$, on fera apparaître la pulsation propre usuelle du système masse ressort, notée $\omega_0$. On prendra comme condition initiale, $\dot{z}(0) = 0$, $z(0) = A$.
    \question En déduire une expression de la période $T$ de cet oscillateur en fonction de $\omega_0$, $A$ et $I = \dfrac{1}{\sqrt{2}} \int_{-1}^1 \dfrac{\dd{u}}{\sqrt{1-u^4}} \sim 1,85$.
    \question Quelle différence par rapport à un oscillateur harmonique ?
\end{questions}

\end{exercise}