% Niveau :      PC *
% Discipline :  Méca
% Mots clés :   Lune

\begin{exercise}{Autour de la Lune}{3}{Spé}
{Mécanique,Mécanique céleste}{lelay}

On s'intéresse à la synchronisation Terre--Lune
\begin{questions}
        \question Que savez-vous des phases de la Lune ? Comment expliquez-vous que de la Terre on ne voit qu'une seule face de la Lune ?
        \question Sans calcul, expliquez pourquoi la Lune ne peut être une sphère parfaite.
        \question En proposant une modélisation simple de la Lune, montrer que sa rotation dans le référentiel géocentrique obéit à une équation bien connue.
        \question Expliquez la stabilisation de la rotation de la lune.
\end{questions}

\end{exercise}

\begin{solution}
\begin{questions}
    \question C LA SYNCHRO
    \question L'assymétrie du champ gravitationnel impose qu'une sphère parfaite n'est pas stable si elle est déformable.
    \question On modélise la Lune par 2 masses reliées par une barre rigide. On calcule la couple au centre de la barre. SI C'EST TROP CHAUD : ne prendre en compte que la force d'inertie d'entraienment et conclure. NE PAS OUBLIER DE FAIRE UNE ANALYSE A PRIORI. Avec la force gravitationnelle, aux petits angles, on trouve (je crois) quelque chose comme $\ddot{\theta} + 4\Omega^2 \sin(\theta) = 0$.
\end{questions}
\end{solution}