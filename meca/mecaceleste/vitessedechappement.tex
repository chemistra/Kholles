% Niveau :      PCSI
% Discipline :  Mécanique céleste
%Mots clés :    Astronaute

\begin{exercise}{Autour des vitesses cosmiques}{1}{Sup}
{Mécanique, Mécanique céleste}{lelay}


\begin{questions}
    \questioncours Donnez la définition de la vitesse d'échappement. Quelle est celle de la Terre ?
    \question Quelle est la distance des satellites géostationnaires par rapport à la Terre ? Quelle est leur vitesse de déplacement ?
    \question Un cosmonaute échoué sur un astéroïde de rayon $R$ et de même densité que la terre parvient a s’en échapper en sautant. Quelle est la valeur maximale de $R$ ?
\end{questions}

\end{exercise}

\begin{solution}

\begin{questions}
    \questioncours $\frac12 m v ^2 - GMm/R = 0 \Rightarrow v =\sqrt{2GM/R} \approx 11$ km/s
    
    \question 3e loi de Kepler $R^3/T^2 = GM/4\pi^2$ d'où avec $T = $24 h, $R = 42300$ km. Vitesse $v = R\omega = 2\pi R/T = \sqrt{GM/R}$ soit 3 km/s.
    
    \question Il faut prendre comme vitesse d'échappement la vitesse maximale d'un être humain (autour de 10 m/s pour les champions olympiques) et en déduire $R$, en sachant que $M = \frac43 \pi \rho_T R^3 = M_T R^3/{R_T}^3$ selon que l'élève connaisse mieux la masse de la Terre ($6\times 10^{24}$ kg) ou sa densité ($5000$ kg/m$^3$). On trouve en OdG 10 km.
\end{questions}

\end{solution}
