\begin{exercise}{Brèves}{1}{Sup}
{Cinématique}{bermu}

Brèves de cinématique. On pourra prendre et justifier les approximations effectuées.

\begin{questions}
    \question Un skieur de 70 kg allant à 8.2 m$\cdot$s$^{-1}$ sur une piste verglacée tombe sur son dos (dans la même direction où il allait avant de tomber). 3 secondes après sa chute, sa vitesse est passée à 3.1 m$\cdot$s$^{-1}$.

    Quand le skieur s'arrète-t-il ? Quel distance aura-t-il parcouru durant sa chute ?

    \question Un couple de patineurs est initialement immobile sur la glace. Se repoussant avec leurs mains, la première personne, qui pèse 68 kg, communique à son ou sa partenaire, qui pèse 52 kg, une vitesse de 10 km/h sur la glace.
    
    Quelle est la vitesse finale des deux patineurs ?
\end{questions}

\end{exercise}

\begin{solution}

\end{solution}
