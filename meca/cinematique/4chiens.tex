\begin{exercise}{Les quatres chiens}{2}{Sup}
{Cinématique}{lelay}

À l'instant $t = 0$, quatre chiens $A$, $B$, $C$ et $D$ initialement situés aux sommets d'un carré de côté $\ell_0$ se mettent en mouvement l'un vers l'autre. $A$ se dirige vers $B$ qui se dirige vers $C$ qui se dirige vers $D$ qui lui-même se dirige vers $A$. Tous les chiens se déplacent à la même vitesse $V$.
\begin{questions}
    \questioncours Donner la définition d'un repère et d'un système de coordonnées. Citer ceux que vous connaissez en expliquant leur fonctionnement.
    \question Faire un schéma. Représenter les positions des chiens ainsi que leurs vecteurs vitesse à $t = 0$, pour $t > 0$ et pour $t = \infty$.
    \question Argumenter que le quadrilatère $ABCD$ reste carré, de côté $\ell(t)$ (on pourra invoquer les symétries de la situation). $\ell(t)$ est-elle une fonction croissante ou décroissante du temps ?
    \question Montrer que $\dv{\ell}{t} = - v$. 
    \question En déduire la durée $t_f$ de la poursuite.
    \question Quelle est la distance $D$ parcourue par chaque chien ?
    \question Déterminer, en coordonnées polaires, l'équation horaire de la trajectoire d'un chien dans un repère à l'origine bien choisie.
\end{questions}

\end{exercise}

\begin{solution}
\begin{questions}
    \questioncours Donner la définition d'un repère et d'un système de coordonnées. Citer ceux que vous connaissez en expliquant leur fonctionnement.
    \question Sa toune
    \question Symétrie SO4, $\ell$ diminue.
    \question Y a plein de manières, On peut écrire $\ell^2 = (\vec{OB} - \vec{OA})^2$, et dériver. 
    \question On a $\dot \ell = - v$ d'où $t_f = \ell_0 / v$, 
    \question Les chiens se déplacent à la vitesse $v$, donc $D = v t_f = \ell_0$
    \question On met le centre au centre. On écrit le vecteur déplacement $\vec{\dd{\ell}}$ de norme $v \dd{t}$ que l'on décompose en deux composantes ($\dd{r}$ sur $\vec{e_r}$ et $r\dd{\theta}$ sur $e_\theta$), d'où $\dot r = \cos \pi/4 v$ et $r\dot\theta = \cos \pi/4 v$ et donc $r(t) = \ell_0 - \frac{\sqrt{2}}{2} vt$ et $\theta(t) =  \int_0^t \frac{\sqrt{2}v}{2r(t)}\dd{t}$
\end{questions}



$t_f = 2/3 \ell_0 / v$, $D = 2 / 3 \ell_0$, en polaire $r = r_0 - \sqrt{3}/2 v t$ et $\theta = V/(2r)$
\end{solution}
,