% Niveau :      PCSI
% Discipline :  Méca

\begin{exercise}{Ressort sur un plan incliné}{1}{Sup}
{Mécanique,Oscillateur harmonique,Ressort}{lelay}

On considère une masse $m$ au bout d'un ressort sans masse de constante de raideur $k$ et de longueur à vide $\ell_0$ qui se déplace sans frottement sur un plan incliné d'un angle $\alpha$ par rapport à l'axe horizontal. Le ressort est fixé en $x=0$.

\begin{questions}
    \question Déterminer la longueur du ressort  à l'équilibre $\ell_\text{éq}$.
    \question On étire le ressort d'une longueur $\Delta\ell$, déterminer le mouvement du ressort. Commenter l'influence de l'angle pente ($\alpha$) sur les oscillations
    \question Retrouver ce résultat par des considérations énergétiques.
\end{questions}
\end{exercise}