% Niveau :      PCSI
% Discipline :  Méca

\begin{exercise}{Assemblage de ressorts}{1}{Sup}
{Mécanique,Oscillateur harmonique,Ressort}{lelay}

\begin{questions}
    \question Soit une masse reliée à un mur par deux ressorts de raideur $k_1$ et $k_2$, et de même longueur à vide $\ell_0$. Donner la raideur $k$ du ressort équivalent.
    \question Soit une masse reliée à un ressort de raideur $k_1$, lui même accroché à un ressort $k_2$ relié à un mur. Donner la raideur $k$ du ressort équivalent.
    \question Soit une masse reliée à un mur par deux ressorts de raideur $k_1$ et $k_2$, et de longueur à vide $\ell_1$ et $\ell_2$. Donner la distance de la masse au mur à l'équilibre en fonction des paramètres du problème.
    \question Soit une masse entre deux murs, reliée à chacun d'eux par un ressort de raideur $k$ et de longueur à vide $l_0$. Donner la position $x(t)$ de la masse en fonction du temps
    \questionbonus Quelle analogie peut-on faire avec l'assemblage de résistances en électronique ?
\end{questions}
\end{exercise}
