% Niveau :      PC
% Discipline :  Méca

\begin{exercise}{Questions préliminaires}{0}{Sup, Spé}
{Oscillateur harmonique,Questions préliminaires}{lelay}

Ces questions font appel soit directement au cours, soit à des raisonnements simples d'ordre de grandeur ou d'analyse dimensionnelle. Elles servent juste à aiguiser le sens physique : pas besoin de les rédiger. Il faut savoir y répondre rapidement.

\begin{questions}
    \question Donner la position à l'équilibre d'une masse $m$ attachée au plafond par un ressort de raiseur $k$. La pulsation des oscillations est-elle la même qui si le système était en position horizontale ?
    \question Donner la force minimale avec laquelle il faut appuyer sur une masse accrochée à un ressort contre un mur pour faire décoller le ressort du mur.
    \question Ordre de grandeur d'une raideur $k$ typique de ressort (ex : ressort de stylo à bille, ou stylo quatre couleurs).
    \question Quelles sont les approximations faites pour assimiler le système masse ressort à un oscillateur harmonique ?
\end{questions}
\end{exercise}