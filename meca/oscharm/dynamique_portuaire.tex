% Niveau :      PCSI
% Discipline :  Méca


\begin{exercise}{Dynamique portuaire à Ploudalmézeau}{2}{Sup}
{Oscillateur harmonique}{lelay}

On considère le port breton de Ploudalmézeau dans le Finistère (29). Les principaux habitants en sont de fiers marins bretons, des goélands et des sardines dont les goélands sont friands.

On note $s(t)$ le nombre de sardines et $g(t)$ le nombre de goélands dans le port. Il y a en permanence de nouvelles sardines qui arrivent dans le port, attirées par le charme naturel des côtes bretonnes. De plus les goélands ont tendance à migrer vers l'intérieur des terres et à ne pas rester dans les environs du port. On modélise la situation comme suit :
\begin{align*}
    \dv{g}{t} &= b\:s(t) - \alpha \\
    \dv{s}{t} &= -c\:g(t) + \beta 
\end{align*}
\begin{questions}
    \question Justifier chacun des termes de la modélisation ci-dessus.
    \question Montrer que $g(t)$ et $s(t)$ obéissent chacun à une équation d'oscillateur harmonique de même pulsation.
    \question Trouver $g(t)$ et $s(t)$ sachant que à $t = 0$ jour, il y a 7 goélands et 250 sardines.
    \question Quel est le déphasage entre $g(t)$ et $s(t)$ ? Justifier qu'elles ne peuvent pas être en phase.
\end{questions}
\textbf{Données :} On pourra supposer que à l'équilibre la population de sardines est de 250 individus et celle de goélands de 5 individus, que chaque goéland mange 10 sardines par jour et qu'un nouveau goéland migre vers l'intérieur des terres chaque jour. 
\end{exercise}