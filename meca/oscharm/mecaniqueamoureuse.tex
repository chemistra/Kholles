% Niveau :      PCSI 
% Discipline :  Méca

\begin{exercise}{\textit{L’amour est enfant de bohème}}{1}{Sup}
{Oscillateur harmonique}{strogatz}

Soit un couple canonique, Roméo et Juliette. On note $R$ l'affection que Roméo porte à Juliette et $J$ l'affection que Juliette porte à Roméo.

\newcommand{\heart}{\ensuremath\heartsuit}
\begin{questions}
    \question \textbf{The story of love and hate.} On considère que plus Roméo aime Juliette, plus celle-ci l'aime en retour. En revanche, plus Juliette s'attache à Roméo, moins il est attiré par elle. On modélise cette situation par
    \begin{align*}
        \dot J &= \alpha R, \\
        \dot R &= -\beta J,
    \end{align*} où $\alpha > 0$ et $\beta >0$.
    \begin{parts}
        \part Justifier le signe de $\alpha$ et $\beta$.
        \part Montrer que $R$ et $J$ vérifient chacun une équation d'oscillateur harmonique de pulsation $\omega_0$. Exprimer $\omega_0$ en fonction de $\alpha$ et $\beta$.
        \part En considérant la situation initiale $R(t=0) = R_0$ et $J(t=0)=0$, résoudre cette équation pour tout $t>0$.
        \part Montrer que la quantité
        $$\heart = \frac12 \alpha R^2+\frac12 \beta J^2$$
        est conservée. À quoi cette quantité vous fait-elle penser ?
        \part Portrait de phase : tracer la trajectoire du vecteur $(J(t), R(t))$ dans le plan $(J, R)$. Expliquer ce qu'il se passe. Donner une interprétation géométrique de $\heart$.
        \part Montrer que Roméo et Juliette ne sont jamais en phase pour le plus grand plaisir des dramaturges shakespeariens.
    \end{parts}
    \question \textbf{Roméo le robot.} On considère maintenant que Roméo aime tout le temps Juliette de la même manière : $\dot R = 0$.
    \begin{parts}
        \part Réécrire le système d'équation et le résoudre.
        \part Que se passe-t-il si $R(t=0) > 0$ ? Si $R(t=0) < 0$ ? 
        \part Celle fois-ci, {\heart} est-elle conservée ?
    \end{parts}
    \question \textbf{Juliette se méfie.} On considère maintenant que Juliette se méfie de cette relation : plus elle aime, plus elle doute. On modélise cela par $$\dot J = \alpha R - \gamma J,$$ avec $\gamma > 0$. 
    \begin{parts}
        \part Comment évolue $\heart$ avec le temps ? A priori, que vaut cette quantité lorsque $t \rightarrow \infty$ ?
        \part Écrivez l'équation vérifiée par $J(t)$, puis celle vérifiée par $R(t)$
        \part (\emph{Si vous avez fait les oscillateurs harmoniques amortis}) Résolvez cette équation et tracez la trajectoire de $(J,R)$ dans le portrait de phase.
    \end{parts}
\end{questions}

\vspace{15mm}

\begin{center}

\begin{minipage}{6cm}
    \itshape L'amour est enfant de bohème \\
    Il n'a jamais, jamais, connu de loi \\
    Si tu ne m'aimes pas, je t'aime \\
    Et si je t'aime, prends garde à toi
    
    \flushright Carmen, \normalfont Georges Bizet, 1875
\end{minipage}
\end{center}
\end{exercise}