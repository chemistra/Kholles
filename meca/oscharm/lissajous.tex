% Niveau :      PCSI
% Discipline :  Méca

\begin{exercise}{La figure de Lissajous}{1}{Sup}
{Mécanique,Oscillateur harmonique,Pendule}{lelay}

Soit un mobile $M$ accroché à un mur par un ressort de raideur $k$ et de longueur à vide $\ell_0$. Le mobile est situé sur un rail et ne peux bouger que selon l'axe $Ox$.
\begin{questions}
    \question Appliquer le PFD à $M$. Donner l'expression générale de la trajectoire $x_M(t)$ de $M$.
    \uplevel{On considère maintenant un autre mobile de masse $m$ relié à $M$ par un ressort $k$, $\ell_0$ dans une direction orthogonale $Oy$. $M$ et $m$ sont reliés de telle manière que à tout moment $x_m(t) = x_M(t)$.}
    \question Faire un schéma. Appliquer le PFD à $m$. Donner l'expression générale de la trajectoire $y_m(t)$ de $m$.
    \question Donner l'expression générale de la position $(x_m(t), y_m(t))$ du mobile $m$. Dessiner la trajectoire de $m$ pour :
    \begin{parts}
        \part $\omega_1 = \omega_2$, $\varphi_1 = \varphi_2$
        \part $\omega_1 = \omega_2$, $\varphi_1 = \varphi_2+\frac\pi2$
        \part $\omega_1 = \omega_2$, $\varphi_1 = \varphi_2+\pi$
        \part $\omega_1 = \omega_2$, $\varphi_1 = \varphi_2+3\frac\pi2$
        \part $\omega_1 = 2\omega_2$, $\varphi_1 = \varphi_2+\frac\pi2$
        \part $\omega_1 = n\omega_2$, $\varphi_1 = \varphi_2+\frac\pi2$, $n\in\mbb{N}$
        \part Que pouvez-vous conjecturer si $\omega_1/\omega_2$ est une fraction rationnelle ? Irrationnelle ?
    \end{parts}
\end{questions}
\end{exercise}