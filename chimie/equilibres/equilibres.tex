% Niveau :      PCSI *
% Discipline :  Chimie Orga
% Mots clés :   Stéréochimie

\begin{exercise}{Dimérisation du dioxyde d'azote}{2}{Sup,Spé}
{Transformationn de la matière,Equilibres chimiques}{bermu}

\begin{questions}
\questioncours Que signifie avoir atteint l'équilibre chimique ? En particulier, que cela signifie-t-il d'un point de vue des réactifs et des produits. \\
La réponse sera appuyée par des arguments de sens chimique et par un critère quantitatif.

\begin{EnvUplevel}
     Le dioxyde d'azote (NO$_2$, un gaz brun) a tendance suivant la température à se dimériser en péroxyde d'azote (N$_2$O$_4$, un gaz limpide). On note $K^\circ$ la constante de cet équilibre dans les conditions standard. 
     
     On suppose initialement que seule une quantité $c_0 = 1$ mol$\cdot$m$^{-3}$ de NO$_2$ est présente dans une enceinte à pression ambiante $P^\circ = 1$ bar et à température ambiante $T_1 = 300$ K.
     
     On introduit $\alpha$, l'avancement relatif comme étant le pourcentage (de volume) de N$_2$O$_4$ par rapport à NO$_2$.
\end{EnvUplevel}

\question \'Ecrire l'équilibre chimique entre NO$_2$ et N$_2$O$_4$, avec comme stoechiométrie 1 pour N$_2$O$_4$. \\
Qu'est-ce qui aurait changé si la stoechiométrie avait été différente ?

\question Exprimer le quotient réactionnel de la réaction $Q_r(\alpha)$ en fonction de $\alpha$ uniquement.

\question (\textsf{Application numérique}) $K^\circ(T_1) = 10^{-1}$. Calculer $\alpha_1$. \\
Justifier que le mélange gazeux soit brun.

\question On refroidit l'enceinte à $T_2 = 200$ K, toutes choses étant égales par ailleurs. Le mélange de gaz devient transparent. Comment évolue donc $K^\circ(T)$ avec $T$ ?

\uplevel{Pour quantifier les choses, il est mesuré avec un colorimètre l'absorption du gaz dans le brun aux températures $T_1$ et $T_2$.

L'absorption passe de $A_1 = 0.12$ à $A_2 = 0.02$.}

\question Trouver $\alpha_2$ et en déduire $K^\circ(T_2)$.


\end{questions}

\end{exercise}

% Niveau :      PCSI *
% Discipline :  Chimie Orga
% Mots clés :   Stéréochimie

\begin{exercise}{Du rose au bleu}{2}{Sup,Spé}
{Transformationn de la matière,Equilibres chimiques}{bermu}

\begin{questions}
\questioncours Critère d'évolution d'une réaction. \\
On introduira formellement toutes les notions présentées.

\begin{EnvUplevel}
     On étudie la réaction des ions cobalt (II) dans l'eau. Après un certain temps, le chlorure de cobalt (II) (CoC$l_4^{2-}$, bleu) réagissent avec l'eau pour former un complexe d'ion cobalt (II) entouré de molécules d'eau (Co(H$_2$O)$_6^{2-}$, rose).
     
     Initialement, on met $n_0 = 10$ mmol de chlorure de cobalt dans $V_0 = 200$ mL d'eau. Le mélange devient rose.
\end{EnvUplevel}

\question Avec ce que vous savez sur la molécule d'eau en tant que solvant, représenter dans l'espace le complexe Co(H$_2$O)$_6^{2-}$. Discuter le nombre de molécules d'eau liées à l'ion cobalt.

\question \'Ecrire l'équation de réaction avec une stoechiométrie 1 pour l'élément cobalt. On note $K^\circ$ la constante de réaction.

\question \'Ecrire le quotient de réaction en fonction de l'avancement $\xi$ (en mol), de $n_0$, de $V$, le volume de la solution et de $C^\circ$, la concentration standard.

\uplevel{On dilue peu a peu la solution, qui redevient bleu lorsque la solution a un volume $V_1 = 450$ mL de solution.}

\question Interpréter.

\question Au vue des observations, estimer $K^\circ$.

\question Quelle quantité de chlorure de sodium faut-il rajouter pour que la solution redevienne rose ?

\end{questions}

\end{exercise}

% Niveau :  PCSI *
% Discipline :  Chimie Orga
% Mots clés :   Stéréochimie

\begin{exercise}{Oxydation du cuivre par l'acide nitrique}{2}{Sup,Spé}
{Transformationn de la matière,Equilibres chimiques}{bermu}

\begin{questions}
\questioncours \'Ecriture de la loi d'action de masse pour différent systèmes. \\
On introduira formellement toutes les notions présentées.

\begin{EnvUplevel}
 On considère la réaction suivante d'oxydation du cuivre par l'acide nitrique :
 $$\mathrm{3 Cu_{(s)} + 8 H^+_{(aq)} + 2 {NO_3^-}_{(aq)} \quad\rightleftharpoons\quad 3 {Cu^{2+}}_{(aq)} + 2 NO_{(g)} + 4 H_20_{(\ell)}}, \qquad K^\circ = 10^{63}.$$
 
 À un instant donné, la solution, de volume $V_0 = 500$ mL, contient 30 mM d’ions Cu$^{2+}$, 80 mM d'ions nitrate NO$_3^-$, et un pH$ = 1$ (pour rappel pH = -log[H$^+$]). Un morceau de cuivre de $m = 12$ g est immergé dans  la solution ($M_\text{Cu} = 63,5$ g$\cdot$mol$^{-1}$). La solution est surmontée d’une atmosphère fermée de volume $V_\text{g} = 1,0$ L où la pression partielle en monoxyde d’azote est $p_\text{NO} = 15$ kPa.
\end{EnvUplevel}

    \question Le système est-il à l'équilibre ?
 
    \question Décrire précisément l’état final du système. \\
 Comment peut on qualifier l'état final (deux réponses attendues).
 
 \uplevel{On refait le protocole avec $m = 1$ g de cuivre.}
 
    \question Refaire la question précédente.
 
    \question En deçà de quelle masse le cuivre est-il limitant ?
 
 
 

\end{questions}


\end{exercise}

\begin{solution}

H$^+$ est limitant. $\xi_\text{max} = 6,25$ mmol. $\epsilon = 4.1\times 10^{-9}$ mol. pH = 8.1.

\end{solution}