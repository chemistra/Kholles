% Niveau :      PCSI *
% Discipline :  Chimie Orga I
% Mots clés :   Spectrométrie UV-visible, Réactions acidobasiques

\begin{exercise}{\emph{Arsenic et vieilles dentelles}}{2}{PCSI}
{Atomistique,Classification périodique, Structure électronique}{bermu}

\begin{questions}
    \questioncours Liaisons covalentes, liaisons hydrogène, liaisons Van der Waals. Analogies et différences. Comparaison des énergies associées.
    
    \uplevel{La famille des pnictogènes est la colonne du tableau périodique dans laquelle se trouve notamment l'arsenic (As) et l'antimoine (Sb).}
    
    \question Les pnictogènes ont une configuration électronique de valence en $n$p$^3$.
    \begin{parts}
        \part Donner la position des pnictogènes dans le tableau périodique.
    
        \part Quels sont les pnictogènes $X$ et $Y$ situés dans les périodes 2 et 3, respectivement ?
        
        \part Quelles sont les propriétés chimiques de ces éléments ?
    \end{parts}

\uplevel{On étudie les hydrures des éléments pnictogènes.}
    \question On s'intéresse tout d'abord à l'hydrure d'arsenic, ou arsine, qui s'écrit de manière générique AsH$_n$. Il est précisé que l'arsine n'est pas une entité radicalaire mais que l'arsenic peut dépasser la règle de l'octet.
    \begin{parts}
    
        \part Suggérer des formules et structures de Lewis probables de l'arsine et préciser leur géométrie attendue suivant la VSEPR.
    
        \part Des techniques d'analyse montrent que les angles H--As--H sont identiques de mesure 92,1$^\circ$. Conclure quant à la structure de l'arsine et commenter cette valeur.
        
        \part Les autres hydrures de pnictogènes ont également cette structure. Justifier rapidement.
        
        \part (\emph{Question facultative}) Comment s'appellent les composés $X$H$_n$ et $Y$H$_n$ ?
\end{parts}

\question On donne dans la table ci-dessous quelques propriétés chimiques de ces hydrures.
\begin{table}[H]
    \centering
    \begin{tabularx}{.7\linewidth}{r|CCCC}
        Période & 2 ($X$) & 3 ($Y$) & 4 (As) & 5 (Sb) \\ \hline\hline
        Température d'ébulition ($^\circ$C) & $-33$ & $-88$ & $-63$ & $-17$ \\ 
        Solubilité (vol. / vol.) & 826 & 0,26 & 0,20 & 0,19 \\ \hline
    \end{tabularx}
    \caption{Comparaisons de propriétés chimiques des hydrures de pnictogènes (CTNP).}
\end{table}

\begin{parts}
    \part Justifier l'évolution de la température d'ébullition des hydrures de pnictogènes observée dans la table ci-dessus.

    \part Les atomes d’arsenic et d’hydrogène ont des électronégativités voisines. Comparer la polarité des molécules $X$H$_n$ et AsH$_n$. Préciser sur un schéma clair l’orientation  du moment dipolaire.
    
    \part En déduire une explication de l'évolution des solubilités des des hydrures de pnictogènes observée dans la table ci-dessus.
    
\end{parts}

\end{questions}

\end{exercise}