% Niveau :      PCSI *
% Discipline :  Chimie Orga I
% Mots clés :   Spectrométrie UV-visible, Réactions acidobasiques

\begin{exercise}{\emph{Arsenic et vieilles dentelles}}{1}{PCSI}
{Atomistique,Classification périodique, Structure électronique}{bermu}

La famille des pnictogènes est la colonne du tableau périodique dans laquelle se trouve notamment l'arsenic (As).

\begin{questions}

    \questioncours VSEPR
    
    \question L'arsenic, de numéro atomique 33, n'a qu'un isotope stable, de nombre de masse 75.
    \begin{parts}
        \part Donner le nombre de protons et le nombre de neutrons de cet isotope et le représenter sous la forme $^{A}_{Z}$As.
    
        \part Quelle est sa structure électronique à l'état fondamental ? En déduire la position de l'arsenic dans le tableau périodique.
    \end{parts}

    \question On étudie  l'hydrure d'arsenic, ou arsine, AsH$_3$.
    \begin{parts}
        \part Donner la structure de Lewis et la géométrie attendue selon la théorie VSEPR de AsH$_3$.
    
        \part L'angle mesuré entre les deux liaisons As--H est de 92,1$^\circ$. Commenter.
        
        \part Les autres hydrures de pnictogènes ont également cette structure. Justifier rapidement ?
    \end{parts}
    
\end{questions}

\end{exercise}