% Niveau :      PCSI *
% Discipline :  Chimie Orga I
% Mots clés :   Spectrométrie UV-visible, Réactions acidobasiques

\begin{exercise}{L'azote dans tout ses états}{1}{PCSI}
{Atomistique,Classification périodique, Structure électronique}{bermu}


\begin{questions}
    \questioncours Liaison délocalisées, formes mésomères et hybrides de résonance.
    
\uplevel{Dans cet exercice, on étudie différentes espèces oxydées de l'azote.}

    \question L'azote existe sous plusieurs formes oxydées : \\
    \begin{tabular}{lll}
        -- le monoxyde d'azote NO ; & -- le dioxyde d'azote NO$_2$ ; & -- l'ion nitronium NO$_2^+$ \\
        & -- l'ion nitrite NO$_2^-$ ;  & -- l'ion nitrate NO$_3^-$.
    \end{tabular}
    
    (il sera recommandé d'organiser sa restitution sous forme de tableau.)
    \begin{parts}
        
        \part Donner les structures de Lewis correspondantes.
        
        \part Donner la géométrie de ces molécules dans la théorie VSEPR et comparer avec les résultats expérimentaux (ci-dessous).
        
        \part Lesquelles de ces molécules ont un moment dipolaire non nul ?
    \end{parts}
    
    \begin{table}[H]
    \centering
    \begin{tabularx}{.7\linewidth}{r|CCCCC}
        Molécule & NO & NO$_2$ & NO$_2^+$ & NO$_2^-$ & NO$_3^{-}$\\ \hline\hline
        Angle O--N--O ($^\circ$) & --- & 134 & 180 & 115 & 120 \\ 
        Longeur N--O (pm) & 115 & 120 & 114 & 124 & 127 \\ \hline
    \end{tabularx}
    \caption{Paramètres géométriques .}
\end{table}
\end{questions}

\end{exercise}