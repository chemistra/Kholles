\begin{exercise}{Glycine}{1}{PCSI}
{Acides et bases,Equilibres chimiques}{lelay}

La glycine (ou glycocolle) est un acide aminé de structure H$_2$N--CH$_2$--COOH.  C'est un ampholyte : en effet, dans l'eau, la glycine est engagée dans deux couples acido-basiques de $\text{p}K_\text{a}$ 2.4 et 9.6.

\begin{questions}
    \question Dans l'eau, la glycite est un zwitterion (ou amphion), c'est à dire qu'elle possède à la fois une charge positive et une charge négative. Donner la représentation de Lewis de la glycine dans l'eau.

    \question Écrire les équilibres acido-basiques faisant intervenir la glycine dans l'eau.

    \question Tracer le diagramme de prédominance des espèces de la glycine en fonction du pH de la solution

    \question En solution neutre, quelle est l'espèce majoritaire ?
    
    \question Quel est le pH approché d'une solution de glycine à 10$^{-1}$~M ?
    
    \question L'électrophorèse est une technique permettant de faire migrer les ions d’une solution, sur un support solide, sous l’action d’un champ électrique. À quel pH faut-il opérer l’électrophorèse d’une solution de glycocolle de concentration 10$^{-1}$~M pour que 99\% des ions migrent vers le pôle positif ?
    
\end{questions}
\end{exercise}

\begin{solution}
\begin{questions}

    \question ion glycinium NH$_3^+$ COOH et ion glycinate NH$_2$ COO$^-$
    
    \question ion glycinium - pH 2.4 - glycine - pH 9.6 - ion glycinate
    
    \question À pH = 7 c'est la glycine, l'amphion.
    
    \question On part d'une concentration $C_0 = 0.1$~mol/L en glycine qui réagit selon 2~glycine~=~glycinium~+~glycinate (réaction prépondérante en solution neutre).
    
    On a $K = Ka_2 / Ka_1 = 10^{-7.2}$ c'est bien l'équilibre de contrôle.
    
    Donc d'après le tableau d'avancement [glycinium] = [glycinate] = $x$ et il se trouve que [H$^+$]$^2$ = $Ka_1 \, Ka_2$ d'où $pH = \frac12(pKa_1 + pKa_2) = 6.2$.
    
    \question On veut [glycinate] = 0.99 $c_0$, on ne prend en compte que l'équilibre glycine - glycinate car l'autre est negligeable. 
    
    On utilise la relation de Henderson : $pH = pKa_2 + log([glycinate][glycine]) = 10.6$.
    
\end{questions}
\end{solution}

%%%%%%%%%%%%%%%%%%%%%%%%%%%%%%%%%%%%%%%%%%%%%%%%%%%%%%%%%%%%%%%%%%%%%%%%%%%%%%%%%%%%%%%%%%%%%%%%%%%%%%%%%%%%%%

\begin{exercise}{Imidazole}{1}{PCSI}
{Acides et bases,Equilibres chimiques}{lelay}

L’imidazole, que l’on notera L pour simplifier est une molécule organique de formule C$_3$N$_2$H$_4$. En solution aqueuse, l’imidazole L se comporte comme une monobase faible, le couple LH$^+$/L ayant un $pK_a$ égal à 6.95.

\begin{questions}

    \question On dispose de 50~mL d'une solution acqueuse d'imidazole de concentration 0.02~mol/L. Quel est le pH de cette solution ?

    \question On ajoute à la solution précédente un volume $V$ d’une solution aqueuse d’acide chlorhydrique de concentration 0,50 mol.L$^{-1}$. Calculer, de manière simplifiée, le pH de la solution obtenue correspondant aux valeurs suivantes de $V$ : 0,5 ; 1,0 ; 1,5 ; 2,0 et 2,5 mL.

    \question Représenter l'allure de la courbe $pH = f(V)$
    
\end{questions}
\end{exercise}

\begin{solution}
\begin{questions}

    \question Reaction L = LH+ + HO-, en la supposant peu avancée $K = x^2 / C_0$, on en déduit 
    $$ pH = \frac12\qty( pK_e + pK_a +\log C_0) = 9.6 $$. On a bien pH > pKa + 1 donc L ets majoritaire et pH > 7.5 donc l'autoprotolyse de l'eau est négligeable.
    
    \question La réaction prépondérante est H+ + L = LH+ de constante 10$^{6.95}$ $\gg 1$ donc quantitative.
    \begin{itemize}
        \item Pour V= 0.5, 1.0 et 1.5 mL, on applique juste henderson en supposant que la réaction est totale et pH = 7.4, 6.95, 6.5
        \item Pour V = 2.0 mL tout l'acide a été consommé, la solution est équivalent a une solution d'acide faible LH+ dans 52 mL de solution, d'où $pH = \frac12(pK_a - \log C) = 4.3$
        
        \item Pour V = 2.5 mL on a un excès de 0.25 mmol de H+ pour 52.5 mL de solution. C'est un mélange d'acide faible et d'acide fort. L'acide fort impose le pH d'où $pH = -log [H30+] = 2.3$
    \end{itemize}
    
    \question Typique titrage de baise faible par acide fort.
    
\end{questions}
\end{solution}

%%%%%%%%%%%%%%%%%%%%%%%%%%%%%%%%%%%%%%%%%%%%%%%%%%%%%%%%%%%%%%%%%%%%%%%%%%%%%%%%%%%%%%%%%%%%%%%%%%%%%%%%%%%%%%

\begin{exercise}{Pluies acides}{1}{Sup}
{Acides et bases,Equilibres chimiques}{bermu}

L’eau de pluie est naturellement acide (pH voisin de 6), en raison du dioxyde de carbone qu’elle
dissout. Cette acidification est très nettement augmentée dans les zones à forte activité industrielle. La
pollution par les oxydes de soufre constitue l’une des hypothèses avancées pour expliquer ce
phénomène.

Pour modéliser l’effet de SO$_2$ sur l’acidité de l’eau, on place de l’eau initialement pure dans un récipient
à l’intérieur duquel est maintenue une pression constante de dioxyde de soufre gazeux égale à $\SI{8e-8}{bar}$ à $T = \SI{298}{K}$.

Pour la commodité des calculs, on considère comme négligeable la concentration de SO$_{2\text{(aq)}}$.

\begin{questions}

    \question Tracer le diagramme de prédominance des espèces acido-basiques du soufre intervenant dans la
solution aqueuse.

    \question Sachant que la solution à l’équilibre est plus acide que l’eau de pluie naturelle, quelle espèce du
diagramme de prédominance précédent est assurément en concentration négligeable ?

    \question En déduire l’équation chimique responsable majoritairement de l’acidification de l’eau.

    \question Calculer alors le pH de la solution
    
    \plusloin Vérifier l’hypothèse formulée en \textsfbf{Q2}.
    
\end{questions}

\paragraph{Données :}
$$\text{p}K_\text{a1}(\mathrm{H_2SO_3/HSO_3^-}) = 1,8  \qquad \text{p}K_\text{a2}(\mathrm{HSO_3^-/SO_{3}^{2-}}) = 7,2.$$
Dissolution de SO$_{2\text{(g)}}$ :

$$\mathrm{SO_{2(g)} + H_2O \longrightarrow H_2SO_3} \qquad K_\textsc{h} = \SI{1,25}{}$$

\end{exercise}

\begin{solution}
\begin{questions}

    \question $\mathrm{H_2SO_3/HSO_3^-/SO_{2}}$
    
    \question $\mathrm{SO_{3}^{2-}}$ négligeable

    \question $\mathrm{SO_{2(g)} + 2 H_2O \longrightarrow HSO_3^- + H_3O^+} \qquad K^\circ = K_\textsc{h} 10^{-\text{p}K_\text{a1}} = 0.02$
    
    \question $K^\circ = \dfrac{\mathrm{[H_3O^+][HSO_3^{-}]}}{P_\mathrm{SO_2}}$ or $\mathrm{[H_3O^+] = [HSO_3^{-}] = 10^{-pH}}$

    $\text{pH} = 4.4$.
    
\end{questions}
\end{solution}