\begin{exercise}{Ressort et physique statistique}{-1}{Spé}
{}{mines}

\paragraph{Question de cours} \textsf{(15 minutes préparation, 10 minutes de passage) \textbf{:}}

Filtrage linéaire en électronique.

\paragraph{Exercice} \textsf{(15 minutes préparation, 15 minutes de passage) \textbf{:}}

On considère une particule de masse $M$ attachée à un ressort, de raideur $k$ et de longeur à vide $\ell$. Elle subit les frottements visqueux de l'air avec un coefficient $\alpha$. On négligera tout frottement solide et action de la gravité.

On lui applique un forçage sinusoïdal d'amplitude $F_0$ et de pulsation $\omega$.

\begin{questions}
    \question Donner l'amplitude des oscillations de la particule. Dans quelles conditions observe-t-on une résonance ? Quelle est alors la pulsation de résonance $\omega_0$ ? Quel est le temps typique du régime transittoire $\tau$ ?
    \uplevel{On suppose maintenant que de petites particules de masse $m$ se déposent sur la grande particule.}
    \question Déterminer une expression approchée (de la forme $\omega_0 - \delta\omega$) de la nouvelle pulsation de résonance quand $n$ petites particules se déposent sur la grande. Quel est l'intérêt d'un tel dispositif ?
    \uplevel{On considère que la grande particule possède $N$ sites où de petites particules peuvent se déposer. On suppose également que l'énergie d'adhésion d'une petite particule est $\varepsilon$.}
    \question Donner la probabilité qu'un site soit occupé. En déduire le nombre moyen de sites occupés, ainsi qu'un ordre de grandeur des fluctuations du nombre de sites occupés autour de cette moyenne.
    \question En déduire une estimée de la pulsation de résonance du système.

\end{questions}

\end{exercise}
