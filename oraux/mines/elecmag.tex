\begin{exercise}{Particule autour d'un fil}{-1}{Spé}
{}{mines}

\paragraph{Question de cours} \textsf{(15 minutes préparation, 10 minutes de passage) \textbf{:}}

Machines thermiques dithermes.

\paragraph{Exercice} \textsf{(15 minutes préparation, 15 minutes de passage) \textbf{:}}

On considère un fil conducteur infini parcouru par un courant $I$ constant. 

\begin{questions}
\question On place à une distance $a$, de part et d'autre symétriquement par rapport à ce fil deux particules de charge $q$, initialement immobiles. Trouver les équations régissant ce problème. Montrer qu'il est possible de se ramener à une équation différentielle ordinaire suivant la coordonnée $r$ uniquement.

\question On place cette fois-ci à une distance $a$ du fil une seule particule de charge $q$ possédant une vitesse initiale $\vec{v_0}=v_0\ve_r$. Quels sont les changements par rapport à la situation précédente ?
\end{questions}


\paragraph{Données :}~\\
\begin{align*}
    \int_{1/e}^{e} \frac{\dd{x}}{\sqrt{1-\ln(x)^2}} \approx 3.98
\end{align*}

\end{exercise}
