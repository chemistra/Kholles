\begin{exercise}{Tube tournant}{-1}{Spé}
{}{mines}

\paragraph{Question de cours} \textsf{(15 minutes préparation, 10 minutes de passage) \textbf{:}}

Équations de Maxwell, forme locale et intégrale.

\paragraph{Exercice} \textsf{(15 minutes préparation, 15 minutes de passage) \textbf{:}}

On considère un tube de section circulaire formant un demi cercle de rayon $R$. Une bille de masse $m$ est placée dans ce tube et peut s'y mouvoir sans frottement. Le tube est mis en rotation autour de son axe de symétrie vertical à la pulsation $\omega$.


\begin{center}
    \begin{tikzpicture}
    
    \draw[->, black!30] (0, -1) -- (0, 3);
    \draw (0,3) node[right=2pt] {$z$};
    
    \draw (-2.5,2) arc(-180:0:2.5);
    \draw (-2.2,2) arc(-180:0:2.2);
    
    \fill[black] (0.5*2.35,2-0.866*2.35) circle (0.15);
    \draw (0.5*2.35,2-0.866*2.35) node[below right=2pt] {$m$};
    
    \draw [xshift=0.3cm,rotate around={180:(-0.4,2.2)},line width=1pt,-stealth] (-0.25,2) arc (-30:210:0.3cm and 0.2cm) node[below right=0.1cm] {$\omega$};
    
    \draw[->, thick] (4, 1.5) -- (4, 0.5);
    \draw (4,1) node[right=2pt] {$\vec{g}$};
    
    \end{tikzpicture}
\end{center}

\begin{questions}
\question Trouver la (ou les) position(s) d'équilibre de la bille en fonction des paramètres du problème.

\question En partant d'une position d'équilibre, quelle quantité minimale d'énergie faut-il fournir à la bille pour qu'elle soit éjectée du tube ?

\question Que se passe-t-il si l'on incline le dispositif par rapport à la verticale ?
\end{questions}

\end{exercise}
