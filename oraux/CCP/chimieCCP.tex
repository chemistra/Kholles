\begin{exercise}{Décomposition du chlorure de sulfuryle}{-1}{Spé}
{Cinétique}{bermudez}

\begin{flushright}
\noindent\textsl{D'après oral CCINP 2022}
\end{flushright}

On considère la réaction de décomposition du chlorure de sulfuryle
$$\mathrm{SO_2C\ell_{2(g)} \longrightarrow SO_{2(g)} + C\ell_{2(g)}}.$$
Celle-ci a lieu dans une enceinte de volume $V$ et à température $T$ maintenus constants. Les gaz sont supposés parfaits. On note la $p_1$ la pression partielle en SO$_2$C$\ell_2$ et $p$ la pression totale. Initialement, il y a uniquement du $\mathrm{SO_2C\ell_{2(g)}}$ dans l'enceinte et la pression est $p_\text{i}$.

\begin{questions}
    \question Montrer que $p_1 = 2p_\text{i} - p$.
    \question Donner la loi horaire $p(t)$.
    \question \`A l'aide des données, montrez que la cinétique est bien d'ordre 1.
    \question Donner le temps de demi-réaction. Quels sont les facteurs susceptibles de modifier celui-ci ?
    \question Déterminer l'énergie d'activation de cette réaction à l'aide des données.
\end{questions}

\paragraph{Données}~

Pression dans le réacteur $p(t)$ en bar en fonction du temps $t$ pour diverses températures :\\[-2.7em]
\begin{center}
\begin{tabular}{l|llllllllllllll}
\hline
$T$ ($^\circ$C) \textbackslash\  $t$ (s) & 0    & 10   & 20   & 30   & 40   & 50   & 60   & 70   & 80   & 90   & 100  & 130  & 170  & 200  \\ \hline
5                           & 1,00 & 1,08 & 1,06 & 1,07 & 1,12 & 1,19 & 1,18 & 1,25 & 1,28 & 1,25 & 1,28 & 1,33 & 1,49 & 1,50 \\
10                          & 1,00 & 1,06 & 1,06 & 1,14 & 1,24 & 1,28 & 1,26 & 1,27 & 1,29 & 1,35 & 1,37 & 1,50 & 1,62 & 1,64 \\
15                          & 1,00 & 1,12 & 1,14 & 1,19 & 1,27 & 1,33 & 1,34 & 1,37 & 1,42 & 1,50 & 1,53 & 1,67 & 1,71 & 1,79 \\
20                          & 1,00 & 1,09 & 1,25 & 1,34 & 1,36 & 1,39 & 1,47 & 1,50 & 1,61 & 1,65 & 1,72 & 1,76 & 1,83 & 1,87 \\
25                          & 1,00 & 1,20 & 1,28 & 1,35 & 1,50 & 1,55 & 1,66 & 1,72 & 1,79 & 1,80 & 1,84 & 1,88 & 1,96 & 1,93 \\\hline\hline
\end{tabular}
\end{center}
\end{exercise}

\begin{exercise}{Le germanium}{-1}{Spé}
{Cristallographie}{bermudez}

\begin{flushright}
\noindent\textsl{D'après oral CCINP 2022}
\end{flushright}

Une maille de germanium, de paramètre de maille $a = \SI{5,651}{\angstrom}$, a une configuration de type diamant : cubique face centrée dont un site tétraèdrique sur deux est occupé.

\begin{questions}
    \question Calculer la coordinence de la maille et la population.
    \question Calculer la compacité de la maille.
    \question Calculer le paramètre de maille et sa masse volumique.
    \question Peut-on loger un atome d'hydrogène ($r_\text{H} = 53$ pm) dans l'autre site tétrahédrique ? Dans le site octahédrique ?
\end{questions}
   

\paragraph{Donnée :} $M_\text{Ge} = \SI{72,64}{g\cdot mol^{-1}}$.
\end{exercise}

% Niveau :      PCSI *
% Discipline :  Chimie Orga I
% Mots clés :   Spectrométrie UV-visible, Réactions acidobasiques

\begin{exercise}{Raffinage du nickel par le procédé Mond}{-1}{PC,MP}
{Thermochimie, Affinité, Déplacement d'équilibre}{bermu}


    Du nickel de très haute pureté peut être obtenu par l’intermédiaire du nickel carbonyle (tétracarbonylenickel) Ni(CO)$_4$. Ce complexe se forme à température modérée et pression ordinaire par action du monoxyde de carbone gazeux CO$_\text{(g)}$ sur des pastilles de nickel Ni$_\text{(s)}$.
    Aucun autre métal n’est susceptible de réagir dans les mêmes conditions.
    
    Après séparation, le nickel carbonyle est décomposé selon la réaction inverse pour donner du métal de haute pureté.

\begin{questions}
    \question \'Ecrire l'équation de réaction, avec une stoechiométrie 1 pour le nickel.
    \question À l’aide des données thermodynamiques fournies, établir, en fonction de la température $T$, les expressions de l’enthalpie libre standard de la réaction, dans les domaines de température 0--43$^\circ$C et 43--200$^\circ$.
    
Pouvait-on prévoir le signe du coefficient de $T$ dans l’expression de l’enthalpie libre standard de réaction ?

    \question Tracer le graphe de l’enthalpie libre standard de réaction $\Delta_\text{r}G^\circ$ en fonction de la
température $T$.

Quelle est la température d’inversion de l’équilibre ?

    \question Quelle est la variance d’un système à l’équilibre constitué de nickel solide, de monoxyde de carbone et de nickel carbonyle gazeux ? Quel est l’effet sur ce système d’une augmentation de pression à température constante ?
    
    \question Quelle est la variance d’un système à l’équilibre constitué de nickel solide, de monoxyde de carbone et de nickel carbonyle liquide ? Quel est l’effet sur ce système d’une augmentation de température à pression constante ?

    \paragraph{Données :} dans les conditions standard.
    
    \'Ebullition de Ni(CO)$_4$ : $T_\text{vap} = 43$ $^\circ$C, $\Delta_\text{vap}H^\circ = 30$ kJ$\cdot$mol$^{-1}$.
    
    Dans les CNTP :
    \begin{center}
\begin{table}[H]
    \qquad\begin{tabular}{r|cccc}
        Espèce & $\mathrm{Ni_{(s)}}$ & $\mathrm{CO_{(g)}}$ & $\mathrm{Ni(CO)_{4 (\ell)}}$ \\ \hline\hline
        $\Delta_\text{f}H^\circ$ (kJ$\cdot$mol$^{-1}$) & --- & $-111$ & $-632$ \\
        $S_m^\circ$ (J$\cdot$mol$^{-1}\cdot$K$^{-1}$) & $30$ & $198$ & $320$ \\ \hline
    \end{tabular}
\end{table}
    \end{center}
\end{questions}

\end{exercise}


% Niveau :      PCSI *
% Discipline :  Chimie Orga I
% Mots clés :   Spectrométrie UV-visible, Réactions acidobasiques

\begin{exercise}{Oxydation du soufre}{-1}{PC,MP}
{Thermochimie, Affinité, Déplacement d'équilibre}{bermu}

    Nous allons nous intéresser au passage du dioxyde de soufre SO$_{2 \text{(g)}}$ au trioxyde de soufre SO$_{3 \text{(g)}}$ par l'action de l'oxygène O$_{2 \text{(g)}}$. Ce passage se fait essentiellement au contact d’un catalyseur spécifique, le pentaoxyde de vanadium V2O5.

\begin{questions}
    \question \'Ecrire l'équation de réaction, avec une stoechiométrie 1 pour le dioxygène.
    
    \question Calculer son enthalpie standard de réaction et son entropie standard de réaction à $T = 300$ K et en déduire l'expression de l’enthalpie libre standard de réaction pour toute température $T$.
    
    \question Quelle est la température d’inversion de l’équilibre ? Préciser l’expression numérique de $\ln K^\circ(T)$ pour toute température ($K^\circ$ désigne la constante d’équilibre).
    
    \uplevel{Les industriels travaillent vers $T = 430$ $^\circ$C sous $p = P^\circ = 1$ bar avec un léger excès de dioxygène
provenant de l’air par rapport à la quantité stoechiométrique 2 SO$_2$ pour 1 O$_2$. Nous allons interpréter ces choix.}

    \question Partons de $\lambda$ moles de dioxygène pur et de $1 - \lambda$ moles de dioxyde de soufre. Dresser un tableau d’avancement et donner la relation liant à l’équilibre le paramètre $\lambda$, l’avancement $\xi$, la constante d’équilibre $K^\circ$ et la pression totale $p$.
    
    \question À $T$ et $p$ fixées, pour quelle valeur de $\lambda$ a-t-on un avancement $\xi$ maximal ?
    
    \paragraph{Données :} dans les CNTP
    \begin{center}
\begin{table}[H]
    \qquad\begin{tabular}{r|cccc}
        Espèce & $\mathrm{SO_{2 (g)}}$ & $\mathrm{SO_{3 (g)}}$ & $\mathrm{O_{2 (g)}}$ \\ \hline\hline
        $\Delta_\text{f}H^\circ$ (kJ$\cdot$mol$^{-1}$) & $-297$ & $-396$ & --- \\
        $S_m^\circ$ (J$\cdot$mol$^{-1}\cdot$K$^{-1}$) & $248$ & $257$ & $205$ \\ \hline
    \end{tabular}
\end{table}
    \end{center}
\end{questions}

\end{exercise}
