\begin{exercise}{Rail de Laplace}{-1}{Spé}
{Induction}{lelay}

On considère un rail de Laplace : dans un champ $\vec{B} = B_0\ve_z$, une barre conductrice est placée en $x_0 > 0$ dans la direction $y$ perpendiculairement à une paire de rails dirigés selon $Ox$, écartés d'une distance $a$ et reliés en $x = 0$ par une résistance $R$.

\begin{questions}
    \uplevel{On donne à la tige une vitesse initiale $v_0$.}
    \question Trouver les équations régissant ce problème. En déduire le comportement de la barre aux temps longs ainsi que le temps caractéristique d'évolution de la vitesse de la barre.
    \question Même question, mais cette fois-ci le système est placé sur un plan incliné d'angle $\alpha$
    \question Que se passe-t-il si on remplace la résistance $R$ par une bobine d'inductance $L$ ?
\end{questions}

\end{exercise}