\begin{exercise}{Cristallographie du germanium}{-1}{Spé}
{Cristallo}{bermudez}

\begin{questions}

    \question  Donner sa configuration électronique du germanium ($Z = 32$) dans son état fondamental. On rappellera les lois qui permettent de donner cette configuration.
    
    \question On s'intéresse à un oxyde du germanium de formule inconnue Ge$_x$O$_y$. Son cristal est tel que les atomes de germanium suivent un arrangement cubique à face centrées alors que les atomes d'oxygène occupent la moitié des sites tétraédriques.
    Donner les valeurs de $x$ et $y$.

    \question Calculer la masse volumique du cristal.
\end{questions}

\paragraph{Données}
\begin{center}
\begin{tabular}{rcc}
    \hline
     & O & Ge \\
    Rayon atomique (pm) & 60 & 125\\
    Nombre de masse & 16 & 72.6 \\ \hline\hline \\
    Numéro atomique & 8 & 32 \\ \hline\hline 
\end{tabular}
\end{center}

\end{exercise}

\begin{solution}

[Ge] = [Ar] 3d$^{10}$ 4s$^2$ 4p$^2$

CFC population de 4, 8 sites tetraedriques (centre des 8 cubes inscrits) donc GeO : $x=1$ et $y=1$.

\end{solution}