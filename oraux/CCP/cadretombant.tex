\begin{exercise}{Cadre tombant}{-1}{Spé}
{Induction,Thermochimie}{lelay}

On considère un cadre métallique de résistance $R$, de largeur $a$ et de hauteur $h$ disposé verticalement dans le plan $yz$. Le point le plus bas du cadre est placé en $z = z_0 > 0$.

Dans l'espace $0 > z > -H$ se trouve un champ magnétique $\vec{B} = B_0\ve_x$ orthogonal au cadre.

À $t=0$ on lâche le cadre, qui tombe sous l'effet de la gravité $g$.

\begin{questions}
    \question En utilisant la loi de Lenz, expliquer ce qu'il se passe lorsque
    \begin{parts}
        \part Le cadre tombe en restant au dessus du champ magnétique
        \part Le cadre entre dans le champ magnétique
        \part Le cadre tombe dans le champ magnétique (en supposant $H > h$)
        \part Le cadre sors du champ magnétique
    \end{parts}
    \question Pour chacun des cas précédents, donner les équations du mouvement puis le temps $t_s$ de sortie du cadre du champ magnétique.
    % \question Faire la même étude pour un cadre...
    % \begin{parts}
    %     \part Carré de côté $a$, incliné de 45 degrés.
    %     \part Circulaire de rayon $a$.
    % \end{parts}
\end{questions}

\end{exercise}

