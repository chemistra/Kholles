% Niveau :      PCSI *
% Discipline :  Chimie Orga
% Mots clés :   Stéréochimie

\begin{exercise}{De la cinétique à l'équilibre}{2}{PCSI}
{Transformationn de la matière,Cinétique,Arrhénius}{bermu}

\begin{questions}
\questioncours \'Evolution de la constante de vitesse chimique avec les concentrations et la température.

\begin{EnvUplevel}
    On étudie la réaction d'isomérisation suivante
    $$\mathrm{\underset{\textsfbf{A}}{{CH_3-CO-CH_3}_{(aq)}} \quad\overset{1}{\underset{-1}{\rightleftharpoons}}\quad \underset{\textsfbf{B}}{{CH_3-C(OH)=CH_2}_{(aq)}}},$$
    pour laquelle la constante d'équilibre est $K^\circ$. On introduit initialement \textsfbf{A} avec une concentration $c_0$.
\end{EnvUplevel}

\question Justifier la qualification d'isomérisation.

\question Quel est l'avancement volumique maximal de la réaction $1$, $\xi_\text{max}$ ?

\question Donner les équations cinétiques des vitesses $v_1$ et $v_{-1}$ des deux réactions 1 et $-1$ en fonction des constantes de vitesses $k_1$ et $k_{-1}$.

\question En déduire un système sous la forme
$$\left\lbrace\mqty{\displaystyle\dv{[\textsfbf{A}]}{t} = f\qty\big([\textsfbf{A}],[\textsfbf{B}])\\[2ex] \displaystyle\dv{[\textsfbf{B}]}{t} = g\qty\big([\textsfbf{A}],[\textsfbf{B}])}\right.$$

\question Justifier qualitativement que la concentration totale est constante et donner cette constante. Montrer que le système d'équation précédent vérifie cette propriété.

\question En déduire que l'on peut séparer le système sous la forme
$$\left\lbrace\mqty{\displaystyle\dv{[\textsfbf{A}]}{t} = -\dfrac{1}{\tau}[\textsfbf{A}] + \text{cte}_1 \\[2ex] \displaystyle\dv{[\textsfbf{B}]}{t} = -\dfrac{1}{\tau}[\textsfbf{B}] + \text{cte}_2}\right.$$
On précisera l'expression de $\tau$, cte$_1$ et cte$_2$. Que représente $\tau$ ?

\question On se place à $t \gg \tau$. Décrire l'état final, et en particulier donner l'avancement volumique final $\xi_\text{f}$, le quotient de réaction final $Q_\text{f}$ et les vitesses de réaction $v_1$ et $v_{-1}$. Commentaire ?

\question En déduire une relation entre $K^\circ$, $k_1$ et $k_{-1}$ et lui donner un sens chimique. 

\question Comment évolue $K^\circ$ en fonction de la température ?



\end{questions}


\end{exercise}