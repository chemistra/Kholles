% Niveau :  PCSI *
% Discipline :  Chimie Orga
% Mots clés :   Stéréochimie

\begin{exercise}{Oxydation du cuivre par l'acide nitrique}{2}{PCSI}
{Transformationn de la matière,Cinétique,Arrhénius}{bermu}

\begin{questions}
\questioncours \'Ecriture de la loi d'action de masse pour différent systèmes. \\
On introduira formellement toutes les notions présentées.

\begin{EnvUplevel}
 On considère la réaction suivante d'oxydation du cuivre par l'acide nitrique :
 $$\mathrm{3 Cu_{(s)} + 8 H^+_{(aq)} + 2 {NO_3^-}_{(aq)} \quad\rightleftharpoons\quad 3 {Cu^{2+}}_{(aq)} + 2 NO_{(g)} + 4 H_20_{(\ell)}}, \qquad K^\circ = 10^{63}.$$
 
 À un instant donné, la solution, de volume $V_0 = 500$ mL, contient 30 mM d’ions Cu$^{2+}$, 80 mM d'ions nitrate NO$_3^-$, et un pH$ = 1$ (pour rappel pH = -log[H$^+$]). Un morceau de cuivre de $m = 12$ g est immergé dans  la solution ($M_\text{Cu} = 63,5$ g$\cdot$mol$^{-1}$). La solution est surmontée d’une atmosphère fermée de volume $V_\text{g} = 1,0$ L où la pression partielle en monoxyde d’azote est $p_\text{NO} = 15$ kPa.
\end{EnvUplevel}

    \question Le système est-il à l'équilibre ?
 
    \question Décrire précisément l’état final du système. \\
 Comment peut on qualifier l'état final (deux réponses attendues).
 
 \uplevel{On refait le protocole avec $m = 1$ g de cuivre.}
 
    \question Refaire la question précédente.
 
    \question En deçà de quelle masse le cuivre est-il limitant ?
 
 
 

\end{questions}


\end{exercise}

\begin{solution}

H$^+$ est limitant. $\xi_\text{max} = 6,25$ mmol. $\epsilon = 4.1\times 10^{-9}$ mol. pH = 8.1.

\end{solution}