% Niveau :      PCSI *
% Discipline :  Chimie Orgam
% Mots clés :   Ballistique, Mécanique du point, PFD, Chute libre

\begin{exercise}{Réaction de Meerwein--Ponndorf--Verley}{3}{PCSI}
{Chimie organique I, AN, SN, E, acétalisation, organométallique}{bermu}

\begin{questions}
\questioncours Préparation expérimentale du bromure de 2-méthylpropylmagnésium \textbf{A}. \\
On détaillera précisément le protocole expérimental avec schémas ainsi que les précautions à prendre et les réactions parasites.

\begin{EnvUplevel}
On se propose d'étudier la séquence réactionnelle qui, partant de la 4,4-diméthylpentan-2-one \textbf{B}
\begin{center}
    \schemestart
        \textbf{B}
        \arrow{->[\textbf{A}][Et$_2$O, reflux]}[,2.1]
        \textbf{C $+$ D}
        \arrow{->[H$^+$][H$_2$O, $0^\circ$C]}[,1.9]
        \textbf{E $+$ D}
    \schemestop\chemnameinit{}~,
    \end{center}
forme l'alcène \textbf{D} ($\mathrm{C_4H_8}$) et l'alcool \textbf{E} ($\mathrm{C_7H_{15}OH}$).
\end{EnvUplevel}

\question Donner la structure de \textbf{C} ainsi que son mécanisme de formation.

\question Déduire la structure de \textbf{E} et commenter les conditions expérimentales de la réaction \textbf{D} $\longrightarrow$ \textbf{E}.

\question Quelles structures sont possibles pour \textbf{D} ? Laquelle vous parait la plus cohérente ?

\uplevel{Zimmerman et Traxler ont proposé d'interpréter la formation de \textbf{D} par un état de transition \textbf{AB} \\ (\textbf{A} + \textbf{B}) ayant une géométrie de cycle cyclohexanique.}

\question En représentant en pointillés les liaisons créées et détruites entre \textbf{A} et \textbf{B}, donner la structure de \textbf{AB}.

\question En déduire le mécanisme concerté \textbf{A} + \textbf{B} $\longrightarrow$ \textbf{C} + \textbf{D}.

\uplevel{Mosher en 1950 a montré que la réaction était énantiosélective en effectuant la séquence réactionnelle précédente en remplacant \textbf{A} par \\
\chemfig{-[1](-[3]Et)-[-1]-[1]MgBr}.}

\question Interpréter ce résultat.


\end{questions}
\end{exercise}