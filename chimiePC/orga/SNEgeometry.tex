% Niveau :      PCSI *
% Discipline :  Chimie Orgam
% Mots clés :   Ballistique, Mécanique du point, PFD, Chute libre

\begin{exercise}{Géométrie de la SN}{2}{PCSI}
{Chimie organique I, SN, E}{poublanc,bermu}

\begin{questions}
\questioncours Stéréochimi

\uplevel{On s'intéresse à présent à la sélectivité de différentes réactions.}
    \question
    \hfill\schemestart[][west]
        \chemname{\chemfig{>:[-3]*6(---(-[-3,,,,Cram2Sides](-[7])-[-1])-(<:[-1]C\ell)--)}}{~}
        \arrow{->[NaOEt][EtOH]}
        \chemname{\chemfig{>:[-3]*6(---(-[-3,,,,Cram2Sides](-[7])-[-1])-=-)}}{$>99\%$}
        \arrow{0}[,0]\+\arrow{0}[,0]
        \chemname{\chemfig{>:[-3]*6(---(-[-3](-[7])-[-1])=--)}}{$<1\%$}
    \schemestop\chemnameinit{}\hfill~
    \paragraph{Aide :} représenter le produit sous forme de chaise.\vspace{1.5em}
\end{questions}

\end{exercise}

    