% Niveau :      PCSI *
% Discipline :  Chimie Orgam
% Mots clés :   Ballistique, Mécanique du point, PFD, Chute libre

\begin{exercise}{Réactions parasites}{2}{PCSI}
{Chimie organique II, SN, E, Activation}{bermu}

\begin{questions}
\questioncours Expliquer la stratégie de synthèse en place dans les réactions suivantes et proposer un mécanisme expliquant que les produits obtenus ne soient pas ceux qui seraient normalement attendus.
\begin{parts}
    \part\hfill
    \schemestart
        \chemname{\chemfig{[:-30]*6(--(-OH)-(-)---)}}{\textsfbf{A}}
        \arrow(.base east--.base west){->[HBr]}
        \chemname{\chemfig{[:-30]*6(---(-[1])(-[-1]Br)---)}}{\textsfbf{A'}}
    \schemestop\chemnameinit{}
    \hfill (R1) \\[1em]
    
    \part\hfill
    \schemestart
        \chemname{\chemfig{-(-[3])(-[-3])-(-[2])-[-2]OH}}{\textsfbf{B}}
        \arrow(.base east--.base west){->[HBr]}
         \chemname{\chemfig{-(-[3]Br)(-[-3])-(-[2])-[-2]}}{\textsfbf{B'}}
    \schemestop\chemnameinit{}
    \hfill (R2) \\
\end{parts}

\question Le MTBE (2-méthoxy-2-méthylpropane, noté \textsfbf{C}) est un additif fréquent des carburants.
\begin{parts}
    \part Une première approche naïve de voie de synthèse du MTBE est la déshydratation d'un mélange équimolaire de méthanol (MeOH) et de 2-méthylpropan-2-ol ($^t$BuOH), mais elle conduit à un nombre important de produits secondaires. \\
    Donner le mécanisme de cette réaction et donner quelques produits secondaires envisageables.
    
    \part Quelle$\cdot$s stratégie$\cdot$s peut-on mettre en place afin de palier à ce problème, en agissant sur les réactifs précédent avant de les mettre en présence ? Préciser succinctement le mécanisme en jeu ainsi que les réactions parasites qui peuvent le concurrencer.
    
    \part Parmi ces stratégies, laquelle donnera le meilleur rendement ?
\end{parts}

\end{questions}

\end{exercise}