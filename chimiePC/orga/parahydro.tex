% Niveau :      PCSI *
% Discipline :  Chimie Orgam
% Mots clés :   Ballistique, Mécanique du point, PFD, Chute libre

\begin{exercise}{Synthèse de l'acide parahydroxybenzoïque}{2}{PCSI}
{Chimie organique II, SN, E, Activation}{bermu}

On s'intéresse à la synthèse de l'acide parahydroxybenzoïque \textsfbf{E} à partir du phénol \textsfbf{A} : \\

\begin{center}
    \schemestart
        \chemname{\chemfig{Br-[3]*6(-=-(-[3]OH)=-=)}}{\textsfbf{A}}
        \arrow{->[\parbox{16mm}{1. NaOH \\ 2. CH$_3$I}][?]}
        \textsfbf{B}
        \arrow{->[Mg][Et$_2$O, refflux]}[0,2]
        \textsfbf{C}
        \arrow{->[\parbox{20mm}{1. CO$_\text{2 (s)}$ \\ 2. H$^+$, H$_2$O}][H$_2$O$_\text{(s)}$]}[0,2.1]
        \textsfbf{D}
        \arrow{->[HI][H$_2$O]}[0,1.5]
        \textsfbf{E}
    \schemestop\chemnameinit{}
\end{center}

\begin{questions}
\question \'Etape \textsfbf{A} $\longrightarrow$ \textsfbf{B}.
\begin{parts}
    \part Le pK$_a$ de \textsfbf{A} est proche de 10, bien plus bas que ceux des couples alcool / alcoolate. \\ Proposer une justification de cette différence.
    \part Donner le nom et le mécanisme de cette réaction et proposer une structure pour le produit \textsfbf{B}.
    \part Préciser l'intérêt synthétique de cette réaction.
    \part Proposer des conditions expérimentales adaptées pour effectuer cette réaction.
\end{parts}

\bigskip

\question \'Etape \textsfbf{B} $\longrightarrow$ \textsfbf{C}
\begin{parts}
    \part Quelle est la nature de cette réaction ? Donner la structure du produit \textsfbf{C}.
    \part Quelle précautions prendre pour cette réaction ?
\end{parts}

\bigskip

\question \'Etape \textsfbf{C} $\longrightarrow$ \textsfbf{D}
\begin{parts}
    \part Donner\,le nom et le mécanisme de cette réaction et proposer une structure pour le produit \textsfbf{D}.
    \part Quel est l'intérêt synthétique de cette réaction ?
\end{parts}

\bigskip

\question \'Etape \textsfbf{D} $\longrightarrow$ \textsfbf{E}
\begin{parts}
    \part Donner le mécanisme de cette réaction et la structure de \textsfbf{E}.
    \part Comment se nomme cette stratégie de synthèse ? L'expliquer en l'illustrant par l'exemple de cette étape.
    \part Quel est l'intérêt synthétique des étapes  \textsfbf{A} $\longrightarrow$ \textsfbf{B} et  \textsfbf{D} $\longrightarrow$ \textsfbf{E} ?
\end{parts}

\end{questions}

\end{exercise}