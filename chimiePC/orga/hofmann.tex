% Niveau :      PCSI *
% Discipline :  Chimie Orgam
% Mots clés :   Ballistique, Mécanique du point, PFD, Chute libre

\begin{exercise}{\'Elimination de Hofmann}{2}{PCSI}
{Chimie organique I, SN, E}{bermu}

\noindent\textsl{On s'intéresse à la sélectivité de différentes réactions sur le 2-bromo-2-méthylbutane.} \\
\textsl{Pour chaque question, on précisera bien le rôle de chaque composé et des conditions expérimentales.}

\begin{questions}
\questioncours Sélectivité$\cdot$s et spécificité$\cdot$s de la réaction d'élimination sur les halogénoalcanes. On illustrera la question de cours entre autres par les résultats expérimentaux suivants :
\begin{center}
    \schemestart
        \chemfig{-[1]-[-1](-[-3]Br)(<:[2])-[1]}
        \arrow{->[NaOEt][EtOH]}
        \chemname{\chemfig{-[1]=[-1](-[-3])-[1]}}{71\%}
        \arrow{0}[,0]\+\arrow{0}[,0]
        \chemname{\chemfig{-[1]-[-1](-[-3])=[1]}}{29\%}
    \schemestop\chemnameinit{}
    
    \schemestart
        \chemfig{-[1]-[-1](-[-3]Br)(<:[2])-[1]}
        \arrow(.mid east--.mid west){->[NaO$^\textit{t}$Bu][$^\textit{t}$BuOH]}
        \chemname{\chemfig{-[1]=[-1](-[-3])-[1]}}{28\%}
        \arrow{0}[,0]\+\arrow{0}[,0]
        \chemname{\chemfig{-[1]-[-1](-[-3])=[1]}}{72\%}
    \schemestop\chemnameinit{}
    \end{center}
    \noindent--$^\textit{t}$Bu (\emph{tert}-butyle) désigne --$\mathrm{(CH_3)_3}$.
    
    \bigskip

\begin{EnvUplevel}
    Une variante du mécanisme précédent consiste à remplacer le brome par une amine tertiaire à l'aide d'une amidure pour augmenter la sélectivité :
    
    \quad\textbf{\'Etape 1 :}\hspace{2em}\schemestart[][-6]
        \chemfig{-[1]-[-1](-[-3]Br)(<:[2])-[1]}
        \arrow{->[NaNH$_2$][THF, $-70^\circ$C]}[,2.2]
        \textbf{A}
    \schemestop\chemnameinit{}\hspace{1.9em}puis lavage sur Büchner et séchage.
    
    \quad\textbf{\'Etape 2 :}\hspace{1.9em}\schemestart[][base west]
        \textbf{A}
        \arrow{->[MeI (3 eq.)][NaHCO$_3$]}[,1.9]
        \textbf{B}
        \arrow{->[AgOH, H$_2$O][$80^\circ$C]}[,2.1]
        \chemname{\chemfig{-[1]=[-1](-[-3])-[1]}}{5\%}
        \arrow{0}[,0]\+\arrow{0}[,0]
        \chemname{\chemfig{-[1]-[-1](-[-3])=[1]}}{95\%}
    \schemestop\chemnameinit{}
    
    THF (tétrahydrofurane) désigne le solvant \chemfig{[:-198,.7]*5(-O----)}\,.
\end{EnvUplevel}
    \question Quelle est la nature de la première étape ? Donner la structure de \textbf{A}. \\
        Justifier le choix de température et de solvant.
    \question Quelle est la nature de la réaction \textbf{A} $\longrightarrow$ \textbf{B} ? Donner la structure de \textbf{B}.
    \question Préciser le rôle de NaHCO$_3$. Quel est l'intérêt de NaHCO$_3$ en particulier ?
    \question Que se passe-t-il lors de la dernière étape ?  Interpréter la sélectivité finale.
    \question Discuter du choix de $\mathrm{I}^\ominus$ et $\mathrm{Ag}^\oplus$.
    
    \question Conclure quant à l'effet des chaînes carbonées sur la sélectivité des réactions des questions précédentes.
\end{questions}

\vfill

\plusloin A. W. von Hofmann,Beiträge zur Kenntniss der flüchtigen organischen Basen, \textit{Annalen der Chemie und Pharmacie}, Vol. 78, No. 3, \textbf{1851},  253--286.

\end{exercise}

\begin{solution}
\begin{center}\small\schemestart
        \chemfig{-[1]-[-1](-[-3]Br)(<:[2])-[1]}
        \arrow{->[NaNH$_2$]}
        \chemname{\chemfig{-[1]-[-1](-[-3]NH_2)(<[2])-[1]}}{\textbf{A}}
        \arrow{->[MeI]}
        \chemname{\chemfig{-[1]-[-1](-[-3]\chemabove{N}{\scriptstyle\hspace{3.5mm}\oplus}(-[0,1.4]\chemabove{\ ,\ I}{\scriptstyle\hspace{5.5mm}\ominus})(-[-3])-[6])(<[2])-[1]}}{\textbf{B}}
        \arrow{->[AgOH]}
        ...
    \schemestop\chemnameinit{}\end{center}
\end{solution}