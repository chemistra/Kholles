% Niveau :      PCSI *
% Discipline :  Chimie Orgam
% Mots clés :   Ballistique, Mécanique du point, PFD, Chute libre

\begin{exercise}{Réactivité du 2-chlorométhylfurane}{2}{PCSI}
{Chimie organique I, SN, SN'{},E}{bermu}

\begin{questions}
\questioncours Aspects cinétiques et thermodynamiques de la concurrence entre les réactions de $\mathrm{S_N1}$, $\mathrm{S_N2}$, et E2. On listera dans un tableau, dans quels cas chaque mécanisme domine.

\begin{EnvUplevel}
    Par la suite, on étudie la réaction du 2-chlorométhylfurane \textbf{A} avec le cyanure de potassium :
    \begin{center}
    \schemestart
        \chemname{\chemfig{[:-18]*5((!\Cnum5)-O(!\Cnum1)-((!\Cnumo{1}2)-[:-24](!\Cnumo{-3}1'{})-[:36]C\ell)=(!\Cnum3)-(!\Cnum4)=(!\Cnum5))}}{\textbf{A}}
        \arrow(.mid east--.mid west){->[KCN][H$_2$O, $40^\circ$C]}[,2]
        \chemname{\chemfig{C_6H_5NO}}{\textbf{B}}
    \schemestop\chemnameinit{}
    \end{center}
\end{EnvUplevel}
\question Proposez une structure pour la molécule \textbf{B}. Quel$\cdot$s mécanisme$\cdot$s peut-on envisager ?

\question Commenter les conditions expérimentales. Quelles précautions faut-il prendre ici ?

\begin{EnvUplevel}
    \vspace{-1.5em}
    \paragraph{Données :} $\mathrm{pK_a}\Big($ \chemfig{N~\chemabove{C}{\scriptstyle\hspace{3.5mm}\ominus}} $\Big/$ \chemfig{N~C-H}$\Big) = 9,2$.
    
    \bigskip
    
    Une étude RMN montre qu'autre produit \textbf{C} est formé en plus de \textbf{B} en proportion 40 : 60. Sur \textbf{C}, le nucléophile s'est substitué sur le C$_5$ de \textbf{A}. Proposer deux mécanismes pour expliquer ce résultat :
\end{EnvUplevel}
    \question l'un s'apparentant à une $\mathrm{S_N1}$, nommé $\mathrm{S_N1}$' ;
    \question l'autre s'apparentant à une $\mathrm{S_N2}$, nommé $\mathrm{S_N2}$'.
    
\uplevel{Il a été montré que la réaction était sous contrôle thermodynamique.}
    \question Expliciter ce que cela signifie et donner, en justifiant, le mécanisme dominant.
    
\begin{EnvUplevel}
    Une étude a été menée cette réaction pour différents substituants sur le carbone 4 de la molécule \textbf{A} :
    \begin{table}[H]
    \centering
    \begin{tabular}{cccc}
        & R$_4$ & \textbf{B} (\%) & \textbf{C} (\%) \\ \hline\hline
       (\textit{i})    &  H    & 40              & 60              \\ \hline
       (\textit{ii})   &  CH(CH$_3$)$_2$     & 16              & 84              \\ 
       (\textit{iii})  &  Br     & 17              & 83             \\  
       (\textit{iv})   &  C(CH$_3$)$_3$     & 25              & 75              \\           
       (\textit{v})   &  CN     & 99              & 1           \\    \hline
    \end{tabular}
    \caption{Proportion des produits \textbf{B} et \textbf{C} formés par la réaction entre \textbf{A} et KCN.}
    \end{table}
\end{EnvUplevel}
    \question Interprétez dans la mesure de vos connaissances ces résultats.
\end{questions}~

\vfill

\plusloin S. Divald \emph{et al.} Reaction of 2,4-chloromethylfurans with Aqueous Potassium Cyanide and Other Nucleophiles, \textit{J. Org. Chem.}, Vol. 41, No. 17, \textbf{1976},  2835--2846.
\end{exercise}