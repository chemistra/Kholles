% Niveau :      PCSI *
% Discipline :  Chimie Orga
% Mots clés :   Stéréochimie

\begin{exercise}{Stéréochimies exotiques}{2}{PCSI}
{Chimie organique I,Stéréochimie}{jbb}

\begin{questions}

\questioncours Rappeler les conditions pour avoir un centre stéréogène et être achiral ?

\question Alors que la plupart des amines trisubstituées (\textsfbf{1}) ne constituent pas des centres stéréogènes, les azyridines (\textsfbf{2}) sont chirales. Expliquer et donner les couples de stéréoisomères (comment les qualifier ?).

\begin{EnvUplevel}
    \centering
    \schemestart
    \chemname{\chemfig{R_1 R_2 R_3 N}}{(\textsfbf{1})}\qquad
    \chemname{\chemfig{Cl-N*3(--(-)(-[4])-)}}{(\textsfbf{2})}
    \schemestop
\end{EnvUplevel}

\question Parfois ce ne sont pas des centres (composé d'un atome) qui créent des dissymétries...
\begin{parts}
    \part Quel exemple connaissez-vous d'élément stéréogènes qui ne sont pas des centres ?
    \part Quel type d'élément stéréogène possède ces trois molécules : le diméthyl-1,3-allène (\textsfbf{3}), le 2,6-dimethylspiro[3.3]heptane (\textsfbf{4}) et le propan-2-ylidène-3-méthylcyclobutane (\textsfbf{5}) ?
\end{parts}

\begin{EnvUplevel}
    \centering
    \schemestart
    \chemname{\chemfig{-[2](-[4]H)=C=(<[-1])(<:[1]H)}}{(\textsfbf{3})}\qquad
    \chemname{\chemfig{[:-45]*4((-[-4])(-[4]H)--([:-45]*4(--(<[-1])(<:[1]H)--))--)}}{(\textsfbf{4})}\qquad
    \chemname{\chemfig{[:-45]*4((-[-4])(-[4]H)--(=(<[-1])(<:[1]H))--)}}{(\textsfbf{5})}
    \schemestop
\end{EnvUplevel}

\question Le trimésytylméthane (\textsfbf{6}) est chiral alors que le triphénylméthane (\textsfbf{7}) et le trimésytylsilane (\textsfbf{8}) ne le sont pas.
\begin{parts}
    \part Expliquer la chiralité du trimésytylméthane (\textsfbf{6}) et illustrer le raisonnement d'un profil d'énergie conformationnelle (approximatif).
    \part Donner les deux configurations du trimésytylméthane (\textsfbf{6}), nommées gauche et droite.
    \part Pourquoi le triphénylméthane (\textsfbf{7}) et le trimésytylsilane (\textsfbf{8}) ne sont pas chiraux ?
    \part Peut-on faire une résolution racémique avec (\textsfbf{6}) ?
    \part\textsf{Bonus :} Ces trois molécules ont une forme acide orangée avec avec un pH $\sim$ 7--9. Expliquer.
\end{parts}

\begin{EnvUplevel}
    \centering
    \schemestart
    \chemname{\chemfig{(-[-1]*6(-(-[,.8])=-(-[,.8])=-(-[,.8])=))(-[3]*6(-(-[,.8])=-(-[,.8])=-(-[,.8])=))(-[7]*6(-(-[,.8])=-(-[,.8])=-(-[,.8])=))}}{(\textsfbf{6})}\qquad
    \chemname{\chemfig{(-[-1]*6(-=-=-=))(-[3]*6(-=-=-=))(-[7]*6(-=-=-=))}}{(\textsfbf{7})}
    \qquad\chemname{\chemfig{SiH(-[-1]*6(-(-[,.8])=-(-[,.8])=-(-[,.8])=))(-[3]*6(-(-[,.8])=-(-[,.8])=-(-[,.8])=))(-[7]*6(-(-[,.8])=-(-[,.8])=-(-[,.8])=))}}{(\textsfbf{8})}
    \schemestop
\end{EnvUplevel}

\question Pour finir, il y a également des plans chiraux comme par exemple le cyclooctène. Expliquer pour cette molécule.

\question Conclure quant à la variété des éléments chiraux dans une molécule ?





\end{questions}


\end{exercise}