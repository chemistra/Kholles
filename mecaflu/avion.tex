% Niveau :      PC
% Discipline :  Mécaflu
%Mots clés :    Viscosité vorticité

\begin{exercise}{Vorticité et aviation civile}{2}{Spé}
{Mécanique des fluides, Fluides réels}{lelay}

\begin{questions}
    \questioncours Définir les différents régimes d'écoulement dans un fluide visqueux. On donnera l'équation de Navier--Stockes et l'expression du nombre de Reynolds $\cal{R}e$.
\begin{EnvUplevel}
On défini la vorticité $\vOm$
$$\vOm = \rot\vv.$$
\end{EnvUplevel}
    \question Quel est le sens physique de la vorticité ? Quelle est l'équation de la dynamique de la vorticité ?
\uplevel{On s'intéresse à la dynamique de la vorticité autour des avions commerciaux au décollage et à l'atterrissage.}
    \question Pourquoi les avions commerciaux sont-ils séparés de quelques minutes au décollage ? Retrouver ce délai à l'aide de l'équation de la vorticité.
\end{questions}

\paragraph{Données :} quelques formules d'analyse vectorielle
\begin{align*}
    (\vv\vdot\grad)\vv &= \grad\qty(\dfrac{\vv^2}{2}) + (\rot\vv)\cross\vv \\
    \rot(\vA\cross\vB) &= (\vB\vdot\grad)\vA - (\vA\vdot\grad)\vB + \vB\,(\div\vA) - \vA\,(\div\vB)
\end{align*}
\end{exercise}