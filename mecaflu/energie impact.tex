% Niveau :      PCSI
% Discipline :  Mécanique céleste
%Mots clés :    Astronaute

\begin{exercise}{Géométrie des barrages}{1}{Sup}
{Mécanique, Mécanique céleste}{lelay}

Un cosmonaute échoué sur un astéroïde de rayon $R$ et de même densité que la terre parvient a s’en échapper en sautant en l’air. 
\begin{questions}
    \question Quelle est la valeur maximale de $R$ ?
\end{questions}

\end{exercise}

\begin{solution}

On estime l'énergie cinétique d'un mec qui saute sur terre ($m g_T h_0 = \frac12 m v_0$), et on regarde quelle taille doit faire un astéroïde pour que sa vitesse d'échappement ce soit $v_0$ ($\frac12 m v_0^2 = G m M_{asteroide} / R$). On trouve qqchose de l'ordre de 2700 m mes calculs sont bons.

\end{solution}
