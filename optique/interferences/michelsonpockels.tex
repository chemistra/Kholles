\begin{exercise}{Modification des franges par effet Pockels}{1}{Spé}
{Interference, Michelson}{lelay}

Un interféromètre de Michelson est réglé en coin d'air d'angle $\alpha$ et éclairé sous incidence quasi-normale par une source de lumière monochromatique de longueur d'onde $\lambda_0 = 589$~nm. Sur l'une des voies de l'interféromètre, on dispose une lable de faible épaisseur $e$ et d'indice $n = 2.3$.

\begin{questions}
    \questioncours Décrire le montage utilisé à l'aide un schéma clair et en expliquant où sont localisées les interférences et comment faire pour les observer.
    \question Quel est l'effet de l'introduction de la lame sur le systèmes de franges ?
    \uplevel{En réalité, la lame est taillée dans un matériau dont l'indice peut varier lorsqu'il est soumis à un champ électrique $E$ intense (effet Pockels). La variation d'indice est de la forme $\Delta n = -\frac{n^3}{2}\ r\ E$, où $E$ est le champ électrique appliqué et $r$ un coefficient de réponse dépendant du matériau. 
    Le matériau utilisé est du nobiate de lithium (LiNbO$_3$) dont le coefficient de réponse est $r = 30$~pm/V.
    On impose à l'aide d'un générateur basse fréquence un champ électrique $E(t)$ oscillant triangulairement dans le temps avec un période $T$ et une amplitude $E_0 = 3$~kV/mm.}
    \question Comment choisir l'épaisseur de la lame pour que l'effet de la modulation soit observable ?
    \question Décrire ce que voit l'observateur.
\end{questions}

\end{exercise}


\begin{solution}

\begin{questions}
    \questioncours Coin d'air,. interf sur les miroirs, franges, rectilignes, une lentille fait l'image des miroirs sur un ecran.
    \question Diff de chemin optique $\delta = 2\ n_\text{air}\ \sin(\alpha)\approx 2\alpha$. Décalage de $\Delta x = -(n-1)e/\alpha$, l'interfrange $i = \lambda_0/2\alpha$ est inchangée.
    \uplevel{En réalité, la lame est taillée dans un matériau dont l'indice peut varier lorsqu'il est soumis à un champ électrique $E$ intense (effet Pockels). La variation d'indice est de la forme $\Delta n = -\frac{n^3}{2}\ r\ E$, où $E$ est le champ électrique appliqué et $r$ un coefficient de réponse dépendant du matériau. 
    Le matériau utilisé est du nobiate de lithium (LiNbO$_3$) dont le coefficient de réponse est $r = 30$~pm/V.
    On impose à l'aide d'un générateur basse fréquence un champ électrique $E(t)$ oscillant triangulairement dans le temps avec un période $T$ et une amplitude $E_0 = 3$~kV/mm.}
    \question On veut $\Delta\Delta x \geq i$ i.e. $e \geq \lambda_0/(2\Delta n)$. On trouve $e$ de l'ordre du mm.
    \question Si $T$ est court les franges sont brouillées. Si $T$ est long, on les voit se tranlater périodiquement
\end{questions}
\end{solution}
