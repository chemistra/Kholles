\begin{exercise}{Couleur d'une émulsion ferrofluide}{1}{Spé}
{Interference, Fabry-Pérot}{lelay}

Une émulsion ferrofluide est composée de gouttelettes de ferrofluide de rayon négligeable, placées dans l'eau d'indice $n = 1.33$. Lorsqu'on soumet le ferrofluide à un champ magnétique, les goutelettes s'organisent en chaînes, la distance entre deux gouttelettes est une constante $d = 220$~nm et toutes les chaînes sont parallèles à une direction $Ox$. On éclaire l'émulsion avec de la lumière blanche se propageant dans la direction $Ox$ et on observe la lumière récupérée dans la direction $-Ox$, due à la diffusion de la lumière par les goutelettes, c'est-à-dire à l'absorption partielle de l'onde incidente et à la réémission d'une onde en phase.

On constante que la lumière est nettement colorée. 

\begin{questions}
    \question Interpréter ces observations
    \question Quelle est la couleur de la solution ?
    \question Lorsqu'on augmente la température, $d$ augmente. Comment alors varie la couleur de la solution ?
\end{questions}

\end{exercise}


\begin{solution}

\begin{questions}
    \question Interf constructive nombreuse à la fabry perot
    \question $\delta = 2nd$, constructif si $\delta = p\lambda_0$. $\lambda_0 = 2nd = 585$~nm, jaune-orangé.
    \question Vers le rouge.
\end{questions}
\end{solution}