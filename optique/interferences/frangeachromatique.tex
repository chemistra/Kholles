\begin{exercise}{Frange achromatique}{2}{Spé}
{Interference, Young}{lelay}

On considère deux fentes d'Young situés à distance $D$ d'un écran, éclairés par une source ponctuelle $S$ monochromatique de longueur d'onde $\lambda_0$ située dans le plan focal objet d'une lentille convergente.

\begin{questions}
    \questioncours Fentes d'Young. Décrire la figure d'interférence observée ainsi que la répartition de l'éclairement $\cal{E}(x)$ sur l'écran.
    \uplevel{Une lame de verre, d'épaisseur $e$ et d'indice $n$ est placée devant une des fentes.}
    \question Déterminer l'expression de l'éclairement sur l'écran dans cette nouvelle situation. Quelle est la nouvelle position de la frange centrale ? De combien d'interfranges s'est-elle déplacée ?
    \uplevel{On remplace désormais la source monochromatique par une source de lumière blanche. L'indice du verre varie avec la longueur d'onde dans le vide selon la loi de Cauchy : 
    $$ n(\lambda) = A + \frac{B}{\lambda^2}$$
    avec $A = 1.489$ et $B = 0.004$ $\mu$m$^2$. On appelle \textit{frange achromatique} celle pour laquelle $\pdv{\Delta\varphi}{\lambda} = 0$ pour $\lambda = \lambda_m = 600$ nm la longueur d'onde moyenne du spectre visible.}
    \question Déterminer la position de la frange achromatique. Donner, en interfrange, l'écart entre la frange achromatique et la frange centrale trouvée à la question précédente.
    \question On veut mesurer l'épaisseur $e$ de la lame en trouvant le déplacement de l'unique frange blanche (qui est la mieux contrastée) dû à l'ajout de la lame. Quelle erreur relative comment-on sur la mesure de $e$ si on considère $n = 1.500$, indépendamment de la longueur d'onde ?
\end{questions}

\end{exercise}