\begin{exercise}{Mesure de l'indice de l'air}{1}{Spé}
{Interference, Michelson}{lelay}

Un interféromètre de Michelson est réglé de manière à observer des franges rectilignes en utilisant une source de lumière monochromatique à $\lambda_0 = 589$ nm. Sur l'une des voies, le faisceau traverse une cuve dont la longueur intérieure est $d = 10.0$ mm. Un détecteur mesure l'intensité de la lumière en un point fixe du champ d'interférences. Initialement, la cuve est vide et le détecteur est placé sur un maximum d'intensité.

On fait entrer de l'air dans la cuve, jusqu'à ce que la pression soit égale à la pression atmosphérique. On voit défiler alternativement 10 franges sombres et 9 franges lumineuses, et le détecteur indique finalement une intensité égale à la moitié de l'intensité maximale.

\begin{questions}
    \questioncours Interféromètre de Michelson
    \question Faire un schéma
    \question Déterminer l'indice de réfraction du gaz dans l'état final.
    \question Comment adapter cette expérience en utilisant des fentes d'Young à la place du Michelson ?
\end{questions}

\end{exercise}