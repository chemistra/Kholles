\begin{exercise}{Mesure d'ondes gravitationnelles}{2}{Spé}
{Interference, Michelson}{lelay}

Les détecteurs d’ondes gravitationnelles LIGO (USA) et VIRGO (Italie) sont des interféromètres à division d'amplitude. Leur structure est celle d’un interféromètre de Michelson dont les bras ont des longueurs effectives notées $L_1$ et $L_2$ de l'ordre de 300~km. . La source de lumière est un laser à $\lambda=1064$~nm d'intensité $I_0 = 125$~W. On note $A_1$ (resp $A_2$) l'amplitude de l'onde lumineuse passant par le bras 1 (resp. 2) au niveau du détecteur.
\begin{questions}
    \questioncours Expliquer le principe de fonctionnement d'un interféromètre de Michelson
    \question Dans chacun des bras de l'interféromètre on maintient un vide poussé ($10^{-6}$~Pa). Pourquoi à votre avis ?
    \question Exprimer la différence de phase $\Delta \phi$ entre $A_1$ et $A_2$ en fonction de $L_1$, $L_2$ et $\lambda$.
    \uplevel{Lorsqu’une onde gravitationnelle traverse l’interféromètre, elle distord l'espace ce qui allonge ou rétrécit légèrement les bras de l'interféromètre. On note donc $L_{1,2} = L^0_{1,2}+\delta L_{1,2}$ où $L_1^0$ et $L_2^0$ sont les longueurs au repos (en l'absence d'ondes gravitationnelles) et $\delta L_1$ et $\delta L_{2}$ reprśentent le changement de longueur des bras. On note $x_0 = L_1^0-L_2^0$ et $\delta x =\delta L_1-\delta L_2$.}
    \question La déformation relative $\delta x / L_1$ est de l'ordre de $10^{-21}$. Quel est l’ordre de grandeur de la variation de $\Delta \phi$ due à une onde gravitationnelle ? Peut–on observer un décalage de franges à l'oeil nu dans ces conditions ?
    \question L'intensité optique $I$ mesurée sur le détecteur est la moyenne temporelle du carré de l'onde lumineuse au niveau du détecteur $A=A_1+A_2$. Montrer que l'on peut exprimer $I$ sous la forme
    $$2 I_0\cos^2(\Delta\phi/2)$$
    \question En supposant $\delta x \ll \lambda$, montrer que l'on peut linéariser la variation d'intensité $\delta I$ sous la forme
    $$
    \delta I = -I_0\frac{4\pi \delta x}{\lambda}\sin(4\pi x_0 /\lambda)
    $$
    \question Pour quelles valeurs de $x$ cette variation intensité est-elle maximale ? En déduire comment est réglé l'interféromètre au repos.
    \question Quelle est l'intensité relative que le détecteur doit résoudre afin de détecter des ondes gravitationnelles ? Est-ce réaliste ?
    \question En réalité on place le détecteur sur une frange sombre. En déduire une nouvelle expression de la variation d'intensité.
    \question Comment faire pour maximiser la variation d'intensité mesurée ?
    
\end{questions}

\end{exercise}

\begin{solution}

\begin{questions}
    \questioncours Expliquer le fonctionnement d'un interféromètre de Michelson
    \question Variations faible pression indice = variation faible indice = cool. On prend donc $n=1$ dans la suite de l'exo   
    \question Trajet dans les deux bras, aller-retour donc $\delta = 2(L_1-L_2)/\lambda$. $\Delta \phi = 2\pi \delta = 4\pi(L_1-L_2)/\lambda$
    \uplevel{Lorsqu’une onde gravitationnelle traverse l’interféromètre, elle distord l'espace ce qui allonge ou rétrécit légèrement les bras de l'interféromètre. On note donc $L_{1,2} = L^0_{1,2}+\delta L_{1,2}$ où $L_1^0$ et $L_2^0$ sont les longueurs au repos (en l'absence d'ondes gravitationnelles) et $\delta L_1$ et $\delta L_{2}$ reprśentent le changement de longueur des bras. On note $x_0 = L_1^0-L_2^0$ et $\delta x =\delta L_1-\delta L_2$.}
    \question $\Delta \phi = 4\pi x_0 /\lambda + 4\pi \delta x/\lambda$. La contribution de l'onde gravitationnelle est le terme en $\delta x$, soit de l'ordre de $10\delta x/\lambda = 10^{-6}$ rad $\ll 1$. À l'oeil nu on peut résoudre le passage d'une frange brillante à une frange sombre, soit une différence de phase de $\pi$. On ne peut donc pas regarder ce décalage des franges directement : c'est un millionnième de ce qu'on mesure en TP.
    \question Interférences à 2 ondes (formule de Fresnel) : $I = A^2= I_0(1+\cos(\Delta \phi)) = I_0(1+ (2\cos^2(\Delta\phi/2)- 1)) = 2I_0\cos^2(\Delta\phi/2)$
    \question Il faut faire un DL à l'ordre 1, puisque $\qty(cos(x)^2)'= -\sin (2x)$, EN SUPPOSANT QUE $x_0$ EST NON NUL (modulo $\pi$)
    \begin{align*}
    \delta I &= -I_0\frac{4\pi \delta x}{\lambda}\sin(4\pi x_0 /\lambda)
    \end{align*}
    \question C'est maximal pour $\phi_0 = \frac{4\pi x_0}{\lambda} = \pi/2$ donc si les signaux sont au repos en quadrature de phase. C'est normal : faire un dessin de graphe de sinus, la plus grande variation est au moment où la dérivée est maximale, donc entre une frange sombre et une frange brillante.
    \question L'écart relatif entre l'intensité mesurée au repos et celle pendant une onde gravitationnelle est $\delta I/2I_0 \sim \delta x/\lambda = 10^{-7}$ : impossible de détecter 0.01 milliwatts de différence sur 100~W... Et en plus on crame le détecteur en lui envoyant 125 W dans la tronche.
    \question Sur une frange sombre, $\sin(4\pi x_0 /\lambda) = 0$ donc il faut pousser à l'ordre 2 le DL. On a
    \begin{align*}
    \delta I &=  -2I_0 \qty(\frac{2\pi \delta x}{\lambda})^2\cos(4\pi x_0 /\lambda) \\
    &=  - 8\pi^2 I_0 \qty(\frac{ \delta x}{\lambda})^2
    \end{align*}
    On est quadratique en $\frac{ \delta x}{\lambda}$ mais en même temps la variation relative d'intensité diverge donc c'est intéressant. Il me semble qu'en réalité c'est un compromis entre les deux qui est adopté.
    \question  Il faut un grand $I_0$ mais pas trop grand pour qu'il soit stable. Il faut aussi un faible lambda mais il faut tout de même un laser puissant... C'est difficile.
    
\end{questions}
\end{solution}