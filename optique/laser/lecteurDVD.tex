\begin{exercise}{Longueur d'onde et waist}{1}{Spé}
{Optique}{lelay}

\begin{questions}
    \questioncours Rappeler la relation entre waist, longueur d'onde et angle de divergence pour un faisceau gaussien.
    
    \question Quel est l'angle de divergence maximal envisageable ? En déduire le waist minimum atteignable par un faisceau gaussien de longueur d'onde $\lambda$.
    
    \question Les lecteurs CD utilisaient un laser infrarouge de longueur d'onde $\lambda = 780$ nm, les lecteurs DVD un laser rouge $\lambda = 650$ nm et les lecteurs Blu-ray un laser violet $\lambda = 405$ nm. Commenter.
    
    \question La capacité d'un DVD est de 4.4 Go, celle d'un disque Blu-ray est 25 Go. La diminution de la longueur d'onde rend-elle compte de cette différence ? 
    
    \question La profondeur du sillon est de 190 nm pour un CD, 160 nm pour un DVD et 100 nm pour un disque Blu-Ray. Qu'est ce qui, à votre avis, justifie ces choix technologiques ?
\end{questions}

\end{exercise}

\begin{solution}

\begin{questions}
    \questioncours $\theta = w_0/z_R = \lambda/\pi w_0$
    
    \question $\theta = \pi/2$, d'où $w_0 = 2 \lambda$
    
    \question Plus c petit plus c petit
    
    \question La taille du spot diminue donc on peut faire des sillons avec des trous plus petits. On peut aussi rapprocher les sillons. Les optiques sont aussi meilleurs, il y a un plus grand nombre d'ouverture, une lentille de plus petite focale...
    
    \question La profondeur est $\lambda/4$ pour avoir des interférences constructives. Donc pas de lumière quand le faisceau tombe dans un puits (0) alors que sur une crête on voit une réflexion (1).
\end{questions}
\end{solution}